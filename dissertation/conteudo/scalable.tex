\chapter{Scalable Learning of Probabilistic Circuits}
\label{ch:scalable}

Considering the many benefits and drawbacks of the current state-of-the-art learning algorithms
addressed in \Cref{ch:learning}, and emphasizing the need for scalability and accessibility, we now
present the main contributions of this dissertation, proposing two novel structure learning
algorithms for probabilistic circuits. We approach the problem of learning scalable PCs from two
distinct points of view. In \Cref{sec:logical}, we are interested in PCs whose support encodes a
given logical constraint as certain knowledge; we show how both the probabilistic issue of data
fitness, as well as the logical question of whether the circuit successfully compiles a knowledge
base can be accomplished by aggregating PC samples into ensembles of models. In \Cref{sec:data}, we
look at PCs solely from the perspective of data fitness; we exploit the connection between PCs and
generative random forests \citep{correia20,ho95} and revisit a well-known technique based on random
projections for constructing random trees \citep{dasgupta08a,dasgupta08b}, presenting a simple and
fast yet effective way of learning PCs.

The contents of \Cref{sec:logical} come from our contributions in \citet{geh21a}, while
\Cref{sec:data} come, in part, from \citet{geh21b}.

\section{A Logical Perspective}
\label{sec:logical}

\Cref{rem:initpc} briefly mentioned the question of compiling logical constraints into smooth,
structure decomposable and deterministic logic circuits (i.e.\ (P)SDDs \cite{darwiche11,kisa14}).
Indeed, although there are many existing approaches to learning circuits from logical formulae,
most are only useful for specific tasks \citep{choi16,choi15,shen17,choi17}. Although there are
more generalistic ways of producing circuits, namely from CNFs and DNFs \citep{oztok15,choi13};
logic formulae which incorporate more complex relationships such as cardinality constraints either
have no tractable representation \citep{nishino16} or require the addition of latent variables
\citep{sinz05}. More importantly, because variables which do not play a role in the logical
formulae are completely discarded in the compilation process, translating these \emph{logic}
circuits into \emph{probabilistic} circuits involves naïve assumptions on the discarded variables,
such as fully factorizing them.

Surprisingly, to our knowledge there have been next to no work on learning the structure of PCs
from scratch by looking at both logical formulae \emph{and} data. Even worse, the couple that do
are restricted to very preliminary work: \citet{mattei19} came up with a \divclass{} prototype for
a top-down approach to sampling a special class of PSDDs whose primes are conjunctions of literals
in a similar manner to \textproc{XPC}s, proposing a Bayesian information criterion to searching the
sample space, yet no practical algorithm was fully formulated; \citet{geh19} expanded on
\citeauthor{mattei19}'s work by formalizing an algorithm and introducing a BDD to guide sampling,
however the generated circuits suffered from an exponential blow-up in size.

In this section, we propose a solution inspired by \citet{geh19,mattei19}, yet without the
previously mentioned problems that come with them. In summary, we propose a sampling procedure to
efficiently generating PSDDs whose primes are always conjunctions of literals; to overcome the
exponential blow-up, these PSDDs only partially encode their prior logical restrictions. To
diversify sampling, local transformations similar in spirit to \incrclass{} algorithms are used.
Of worth, we found that not only is this process incredibly fast even under intricate logical
formulae, but by combining samples into an ensemble we achieve competitive results against the
state-of-the-art.

Before we address our contributions, we should first fix some notation on the issue of
propositional logic. We treat propositional variables as 0/1-valued random variables and use them
interchangeably. Given a Boolean formula, we write $\langle f\rangle$ to denote its semantics,
i.e.\ the Boolean function represented by $f$. For Boolean formulas $f$ and $g$, we write $f\equiv
g$ if they are logically equivalent, i.e.\ if $\langle f\rangle=\langle g\rangle$; we abuse
notation and write $\phi\equiv f$ to indicate that $\phi=\langle f\rangle$ for a Boolean function
$\phi$. We overload the scope function once again for logical formulae: $\Sc(f)$ denotes the set of
variables that appear in $f$.

Because we are only interested in smooth, structure decomposable and deterministic PCs whose
support is defined by a logical formula, we shall adopt the usual notation of PSDDs, which we
present next.

\begin{definition}[Partition]
  Let $\phi(\set{x},\set{y})$ be a Boolean function over disjoint sets of variables $\set{X}$ and
  $\set{Y}$, and $\mathcal{D}=\{(p_i,s_i)\}_{i=1}^k$ be a set of tuples where $p_i$ (the prime) and
  $s_i$ (the sub) are formulae over $\set{X}$ and $\set{Y}$ respectively, satisfying $p_i\wedge
  p_j\equiv\bot$ for each $i\neq j$ and $\bigvee_{i=1}^k p_i\equiv\top$. We say that $\mathcal{D}$
  is an ($\set{X}$, $\set{Y}$)-partition of $\phi$ if and only if
  $\phi\equiv\bigvee_{i=1}^k(p_i\wedge s_i)$.
\end{definition}

An (exact) partition\footnote{The naming \emph{partition} is unfortunate. The nomenclature in
probabilistic circuits is full of many other partitions, either using the term to conjure meaning
from set theory when dealing with data splits (see \Cref{sec:divconq}), partition nodes in region
graphs (see \Cref{sec:random}) or in PSDD literature in this section. Here (and only here),
partitions will mean stricly the latter.} is no more than a smooth, structure decomposable and
deterministic circuit rooted at a sum (or disjunction) node whose children are products (or
conjunctions); the primes of these products must necessarily be mutual exclusive (formally,
$p_i\wedge p_j\equiv\bot$) and exhaustive (formally, $\bigvee_{i=1}^k p_i\equiv\top$).
Semantically, a partition states that a logical formula decomposes into $k$ exact conjunctions of
pairs of prime and sub.

\subsection{\textproc{SamplePSDD}}
\label{subsection:samplepsdd}

\subsection{Experiments}

\section{A Data Perspective}
\label{sec:data}

\subsection{\textproc{LearnRP}}

\subsection{Experiments}
