%!TeX root=../tese.tex

\chapter{Appendix}

\section{Proofs}

\begin{theorem}
  \label{thm:summix}
  Let $\mathcal{C}$ a probabilistic circuit whose first $l$ layers are composed solely of sum
  nodes. Call $\Nodes$ the set of all nodes in layer $l+1$. $\mathcal{C}$ is equivalent to a PC
  $\mathcal{C}'$ whose root is a sum node with $\Nodes$ as children.
\end{theorem}
\begin{proof}
  We adapt a similar proof due to \citet{jaini18b}. Every sum node is of the form
  \begin{equation*}
    \Sum(\set{x})=\sum_{\Child\in\Ch(\Sum)}w_{\Sum,\Child}\cdot\Child(\set{x}).
  \end{equation*}
  Particularly, every child $\Child$ in a sum node in layer $1\leq i\leq l-1$, is a sum node, and
  so for the first layer we have that
  \begin{align*}
    \Sum(\set{x})&=\sum_{\Child_1\in\Ch(\Sum)}w_{\Sum,\Child_1}\sum_{\Child_2\in\Ch(\Child_1)}
    w_{\Child_1,\Child_2}\Child_2(\set{x})\\
                 &=\sum_{\Child_1\in\Ch(\Sum)}\sum_{\Child_2\in\Ch(\Child_1)}w_{\Sum,\Child_1}
    w_{\Child_1,\Child_2}\Child_2(\set{x}).
  \end{align*}
  Define a one-to-one mapping that takes a tuple $(\Child_1,\Child_2)$ where $\Child_1\in\Ch(\Sum)$
  and $\Child_2\in\Ch(\Child_1)$ and returns a (unique) path from $\Sum$ to every grandchild
  $\Child_2$ of $\Sum$. Call $\set{K}$ the set of all paths, and $w_{\Sum,\Child_1}$ and
    $w_{\Child_1,\Child_2}$ the weights for one such path. We can merge these two weights into a
  single weight $w_{\Sum,\Child_2}'= w_{\Sum,\Child_1}\cdot w_{\Child_1,\Child_2}$, yielding
  \begin{equation*}
    \Sum(\set{x})=\sum_{(w_{\Sum,\Child_1},w_{\Child_1,\Child_2})\in\set{K}} w_{\Sum,\Child_2}'
    \Child_2(\set{x}).
  \end{equation*}
  This ensures that two consecutive sum layers can be collapsed into a single layer. Particularly,
  for the first (root) and second layers, the above transformation generates a circuit with one
  fewer layer and whose root has $\bigo(nm)$ edges, where $n$ and $m$ are the number of edges coming
  from the original root and its children respectively. We can apply this procedure until there are
  no more consecutive sum nodes. This results in a PC of the form
  \begin{equation*}
    \Sum(\set{x})=\sum_{\Child\in\Ch(S)} w_{\Sum,\Child}\Node(\set{x}),
  \end{equation*}
  where $\Node\in\Nodes$. The number of children of the resulting root sum node will be exponential
  on the number of edges of its children.
\end{proof}
