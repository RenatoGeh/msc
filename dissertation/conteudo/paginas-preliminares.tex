%!TeX root=../tese.tex
%("dica" para o editor de texto: este arquivo é parte de um documento maior)
% para saber mais: https://tex.stackexchange.com/q/78101/183146

%%%%%%%%%%%%%%%%%%%% DEDICATÓRIA, RESUMO, AGRADECIMENTOS %%%%%%%%%%%%%%%%%%%%%%%

% Reinicia o contador de páginas (a próxima página recebe o número "i") para
% que a página da dedicatória não seja contada.
\pagenumbering{roman}

% Agradecimentos:
% Se o candidato não quer fazer agradecimentos, deve simplesmente eliminar
% esta página. A epígrafe, obviamente, é opcional; é possível colocar
% epígrafes em todos os capítulos. O comando "\chapter*" faz esta seção
% não ser incluída no sumário.
\chapter*{Acknowledgements}

% Resumo e abstract são definidos no arquivo "metadados.tex". Este
% comando também gera automaticamente a referência para o próprio
% documento, conforme as normas sugeridas da USP.
\printResumoAbstract


%%%%%%%%%%%%%%%%%%%%%%%%%%% LISTAS DE FIGURAS ETC. %%%%%%%%%%%%%%%%%%%%%%%%%%%%%

% Como as listas que se seguem podem não incluir uma quebra de página
% obrigatória, inserimos uma quebra manualmente aqui.
\makeatletter
\if@openright\cleardoublepage\else\clearpage\fi
\makeatother

% Todas as listas são opcionais; Usando "\chapter*" elas não são incluídas
% no sumário. As listas geradas automaticamente também não são incluídas
% por conta das opções "notlot" e "notlof" que usamos mais acima.

% Normalmente, "\chapter*" faz o novo capítulo iniciar em uma nova página, e as
% listas geradas automaticamente também por padrão ficam em páginas separadas.
% Como cada uma destas listas é muito curta, não faz muito sentido fazer isso
% aqui, então usamos este comando para desabilitar essas quebras de página.
% Se você preferir, comente as linhas com esse comando e des-comente as linhas
% sem ele para criar as listas em páginas separadas. Observe que você também
% pode inserir quebras de página manualmente (com \clearpage, veja o exemplo
% mais abaixo).
\newcommand\disablenewpage[1]{{\let\clearpage\par\let\cleardoublepage\par #1}}

% Nestas listas, é melhor usar "raggedbottom" (veja basics.tex). Colocamos
% a opção correspondente e as listas dentro de um grupo para ativar
% raggedbottom apenas temporariamente.
\bgroup
\raggedbottom

%%%%% Listas criadas manualmente

\chapter*{\raggedright{}Nomenclature}

%\chapter*{Lista de Símbolos}
\disablenewpage{\section*{\raggedright\Large\textbf{List of Symbols}}}

\begin{tabular}{rl}
        $X$, $Y$, $Z$, $\ldots$ & Random variables or propositional variables\\
        $x$, $y$, $z$, $\ldots$ & Assignments of random or propositional variables\\
        $\set{X}$, $\set{Y}$, $\set{Z}$, $\ldots$ & Sets of variables\\
        $\set{x}$, $\set{y}$, $\set{z}$, $\ldots$ & Sets of assignments\\
        $\mathcal{X}$, $\mathcal{Y}$, $\mathcal{Z}$, $\ldots$ & Sample space of random variables\\
        $\indep$ & Statistical independence\\
        $\langle f\rangle$ & Semantics of Boolean formula $f$\\
        $f\equiv g$ & Equivalence between Boolean formulae $f$ and $g$ (i.e.\ $\langle f\rangle = \langle g\rangle$)\\
        $\left[a..b\right]$ & Integer set $\{a,a+1,\ldots,b\}\subset\mathbb{Z}$ for $b\geq a$\\
        $\left[b\right]$ & Integer set $\{1,2,\ldots,b\}\subset\mathbb{Z}$ for $b>0$\\
        $\liv\phi\riv$ & Iverson bracket (i.e.\ 1 if $\phi$ is true, 0 otherwise)\\
        $\Node$, $\Sum$, $\Prod$, $\Leaf$ & Graph nodes\\
        $\Nodes$, $\Sums$, $\Prods$, $\Leaves$ & Sets of nodes\\
        $\Ch(\Node)$ & Set of all children of node $\Node$\\
        $\Pa(\Node)$ & Set of all parents of node $\Node$\\
        $\Desc(\Node)$ & Set of all descendants of node $\Node$\\
        $\Sc(\Node)$ & Scope of node $\Node$\\
        $\operatorname{Inputs}(\mathcal{C})$ & Set of input nodes of circuit $\mathcal{C}$\\
        $\mathcal{N}$ & Gaussian distribution\\
        $v^\gets$, $v^\to$ & Left and right children of vtree node $v$\\

\end{tabular}

% Quebra de página manual
%\clearpage

%%%%% Listas criadas automaticamente

% Você pode escolher se quer ou não permitir a quebra de página
%\listoffigures
%\renewcommand{\listfigurename}{\Large\textbf{List of Figures}}
\disablenewpage{\section*{\raggedright\Large\textbf{List of Figures}}}
\disablenewpage{\listoffigures}

% Você pode escolher se quer ou não permitir a quebra de página
%\listoftables
%\renewcommand{\listtablename}{\Large List of Tables}
%\disablenewpage{\listoftables}

% Esta lista é criada "automaticamente" pela package float quando
% definimos o novo tipo de float "program" (em utils.tex)
% Você pode escolher se quer ou não permitir a quebra de página
%\listof{program}{\programlistname}
%\disablenewpage{\listof{program}{\programlistname}}

%\renewcommand{\listalgorithmname}{\Large\textbf{List of Algorithms}}
\disablenewpage{\section*{\raggedright\Large\textbf{List of Algorithms}}}
\disablenewpage{\listofalgorithms}

\tcblistof[\section*]{eg}{\Large List of Examples}

\tcblistof[\section*]{rem}{\Large List of Remarks}

\section*{Notation}

We use the following notation throughout the work. Random variables are written in upper case (e.g.
$X$, $Y$) and their values in lower case (e.g. $x$, $y$). We write $\mathcal{X}$ as the sample
space and for independence we use $\indep$ to indicate that two variables $X$ and $Y$ are
statistically independent, i.e. $X\indep Y$. We identify propositional variables with 0/1-valued
random variables, and use them interchangeably.  Sets of variables and their joint values are
written in boldface (e.g. $\set{X}$, $\set{x}$). Given a Boolean formula $f$, we write $\langle
f\rangle$ to denote its semantics, i.e. the Boolean function represented by $f$. For Boolean
formulas $f$ and $g$, we write $f\equiv g$ if they are logically equivalent, that is, if $\langle
f\rangle = \langle g\rangle$; we abuse notation and write $\phi\equiv f$ to indicate that $\phi=
\langle f\rangle$ for a Boolean function $\phi$. We use the notation $[a..b]$, with $b\geq
a$ to denote the integer set $\{a,a+1,\ldots,b\}\subset\mathbb{Z}$. Similarly, we use $[b]$ as an
equivalent for $[1..b]$. We denote the Iverson bracket as $\liv\phi\riv$, i.e.\ a function that
returns 1 if $\phi$ is true and 0 otherwise. In the context of graph theory, we use sans serif
letters for graph nodes (e.g. $\Node$, $\Sum$, $\Prod$, $\Leaf$) and bold variants for sets of
nodes (e.g. $\Nodes$, $\Sums$, $\Prods$, $\Leaves$). We call $\Ch(\Node)$ the set of all children
of a node $\Node$, $\Pa(\Node)$ as the set of all parents, and $\Desc(\Node)$ the set of all
descendants.

% Sumário (obrigatório)
\tableofcontents

\egroup % Final de "raggedbottom"

% Referências indiretas ("x", veja "y") para o índice remissivo (opcionais,
% pois o índice é opcional). É comum colocar esses itens no final do documento,
% junto com o comando \printindex, mas em alguns casos isso torna necessário
% executar texindy (ou makeindex) mais de uma vez, então colocar aqui é melhor.
%\index{Inglês|see{Língua estrangeira}}
%\index{Figuras|see{Floats}}
%\index{Tabelas|see{Floats}}
%\index{Código-fonte|see{Floats}}
%\index{Subcaptions|see{Subfiguras}}
%\index{Sublegendas|see{Subfiguras}}
%\index{Equações|see{Modo Matemático}}
%\index{Fórmulas|see{Modo Matemático}}
%\index{Rodapé, notas|see{Notas de rodapé}}
%\index{Captions|see{Legendas}}
%\index{Versão original|see{Tese/Dissertação, versões}}
%\index{Versão corrigida|see{Tese/Dissertação, versões}}
%\index{Palavras estrangeiras|see{Língua estrangeira}}
%\index{Floats!Algoritmo|see{Floats, Ordem}}

