%!TeX root=../tese.tex
%("dica" para o editor de texto: este arquivo é parte de um documento maior)
% para saber mais: https://tex.stackexchange.com/q/78101/183146

%%%%%%%%%%%%%%%%%%%%%%%%%%%%%%%%%%%%%%%%%%%%%%%%%%%%%%%%%%%%%%%%%%%%%%%%%%%%%%%%
%%%%%%%%%%%%%%%%%%%%%%%%%%%%% METADADOS DA TESE %%%%%%%%%%%%%%%%%%%%%%%%%%%%%%%%
%%%%%%%%%%%%%%%%%%%%%%%%%%%%%%%%%%%%%%%%%%%%%%%%%%%%%%%%%%%%%%%%%%%%%%%%%%%%%%%%

% Estes comandos definem o título e autoria do trabalho e devem sempre ser
% definidos, pois além de serem utilizados para criar a capa, também são
% armazenados nos metadados do PDF.
\title{
    % Obrigatório nas duas línguas
    titlept={Aprendizado Escalável de Circuitos Probabilísticos},
    titleen={Scalable Learning of Probabilistic Circuits},
    % Opcional, mas se houver deve existir nas duas línguas
    %subtitlept={},
    %subtitleen={},
    % Opcional, para o cabeçalho das páginas
    %shorttitle={Título curto},
}

\author[fem]{Renato Lui Geh}

% Para TCCs, este comando define o supervisor
\orientador[fem]{Prof. Denis Deratani Mauá, PhD}

% Se não houver, remova; se houver mais de um, basta
% repetir o comando quantas vezes forem necessárias
%\coorientador{Prof. Dr. Ciclano de Tal}
%\coorientador[fem]{Profª. Drª. Beltrana de Tal}

% A página de rosto da versão para depósito (ou seja, a versão final
% antes da defesa) deve ser diferente da página de rosto da versão
% definitiva (ou seja, a versão final após a incorporação das sugestões
% da banca).
\defesa{
  nivel=mestrado, % mestrado, doutorado ou tcc
  % É a versão para defesa ou a versão definitiva?
  %definitiva,
  % É qualificação?
  %quali,
  programa={Computer Science},
  membrobanca={Profª. Drª. Fulana de Tal (orientadora) -- IME-USP [sem ponto final]},
  % Em inglês, não há o "ª"
  %membrobanca{Prof. Dr. Fulana de Tal (advisor) -- IME-USP [sem ponto final]},
  membrobanca={Prof. Dr. Ciclano de Tal -- IME-USP [sem ponto final]},
  membrobanca={Profª. Drª. Convidada de Tal -- IMPA [sem ponto final]},
  % Se não houver, remova
  apoio={This work was supported by CNPq grant \#133787/2019-2, CAPES grant \#88887.339583/2019-00
  and EPECLIN FM-USP.},
  local={São Paulo},
  data=2021-11-1, % YYYY-MM-DD
  % Se quiser estabelecer regras diferentes, converse com seu
  % orientador
  direitos={I hereby authorize the total or partial reproduction and publishing of this work for
    educational ou research purposes, as long as properly cited.}
  % Isto deve ser preparado em conjunto com o bibliotecário
  %fichacatalografica={nome do autor, título, etc.},
}

% As palavras-chave são obrigatórias, em português e
% em inglês. Acrescente quantas forem necessárias.
\palavrachave{Circuitos probabilísticos}
\palavrachave{Aprendizado de máquina}
\palavrachave{Modelos probabilísticos}

\keyword{Probabilistic circuits}
\keyword{Machine learning}
\keyword{Probabilistic models}

% O resumo é obrigatório, em português e inglês.
\resumo{
}

\abstract{
}
