%!TeX root=../tese.tex
%("dica" para o editor de texto: este arquivo é parte de um documento maior)
% para saber mais: https://tex.stackexchange.com/q/78101/183146

%%%%%%%%%%%%%%%%%%%%%%%%%%%%%%%%%%%%%%%%%%%%%%%%%%%%%%%%%%%%%%%%%%%%%%%%%%%%%%%%
%%%%%%%%%%%%%%%%%%%%%%%%%%%%% METADADOS DA TESE %%%%%%%%%%%%%%%%%%%%%%%%%%%%%%%%
%%%%%%%%%%%%%%%%%%%%%%%%%%%%%%%%%%%%%%%%%%%%%%%%%%%%%%%%%%%%%%%%%%%%%%%%%%%%%%%%

% Estes comandos definem o título e autoria do trabalho e devem sempre ser
% definidos, pois além de serem utilizados para criar a capa, também são
% armazenados nos metadados do PDF.
\title{
    % Obrigatório nas duas línguas
    titlept={Aprendizado Escalável de Circuitos Probabilísticos},
    titleen={Scalable Learning of Probabilistic Circuits},
    % Opcional, mas se houver deve existir nas duas línguas
    %subtitlept={},
    %subtitleen={},
    % Opcional, para o cabeçalho das páginas
    %shorttitle={Título curto},
}

\author[masc]{Renato Lui Geh}

% Para TCCs, este comando define o supervisor
\orientador[masc]{Professor Denis Deratani Mauá}

% Se não houver, remova; se houver mais de um, basta
% repetir o comando quantas vezes forem necessárias
%\coorientador{Prof. Dr. Ciclano de Tal}
%\coorientador[fem]{Profª. Drª. Beltrana de Tal}

% A página de rosto da versão para depósito (ou seja, a versão final
% antes da defesa) deve ser diferente da página de rosto da versão
% definitiva (ou seja, a versão final após a incorporação das sugestões
% da banca).
\defesa{
  nivel=mestrado, % mestrado, doutorado ou tcc
  % É a versão para defesa ou a versão definitiva?
  %definitiva,
  % É qualificação?
  %quali,
  programa={Computer Science},
  membrobanca={Professor Denis Deratani Mauá (Chair) -- University of São Paulo},
  membrobanca={Professor Guy Van den Broeck -- University of California, Los Angeles},
  membrobanca={Professor Bryon Aragam -- University of Chicago},
  % Se não houver, remova
  apoio={This work was supported by CNPq grant \#133787/2019-2, CAPES grant \#88887.339583/2019-00
  and EPECLIN FM-USP.},
  local={São Paulo},
  data=2021-11-1, % YYYY-MM-DD
  % Se quiser estabelecer regras diferentes, converse com seu
  % orientador
  direitos={I hereby authorize the total or partial reproduction and publishing of this work for
    educational ou research purposes, as long as properly cited.}
  % Isto deve ser preparado em conjunto com o bibliotecário
  %fichacatalografica={nome do autor, título, etc.},
}

% As palavras-chave são obrigatórias, em português e
% em inglês. Acrescente quantas forem necessárias.
\palavrachave{Circuitos probabilísticos}
\palavrachave{Aprendizado de máquina}
\palavrachave{Modelos probabilísticos}

\keyword{Probabilistic circuits}
\keyword{Machine learning}
\keyword{Probabilistic models}

% O resumo é obrigatório, em português e inglês.
\abstract{
  The rising popularity of generative models together with the growing need for flexible and exact
  inferences has motivated the machine learning community to look for expressive yet tractable
  probabilistic models. Probabilistic circuits (PCs) are a family of tractable probabilistic models
  capable of answering a wide range of queries exactly and in polynomial time. Their operational
  syntax in the form of a computational graph and their principled probabilistic semantics allow
  their parameters to be estimated by the highly scalable and efficient optimization techniques
  used in deep learning. Importantly, tractability is tightly linked to constraints on their
  underlying graph: by enforcing certain structural assumptions, queries like marginals,
  \emph{maximum a posteriori} or entropy become linear time computable while still retaining great
  expressivity. While inference is usually straightforward, learning PCs that both obey the needed
  structural restrictions and exploit their expressive power has proven a challenge. Current
  state-of-the-art structure learning algorithms for PCs can be roughly divided into three main
  categories. Most learning algorithms seek to generate a usually tree-shaped circuit from
  recursive decompositions on data, often through clustering and costly statistical (in)dependence
  tests, which can become prohibitive in higher dimensional data. Alternatively, other approaches
  involve constructing an intricate network by growing an initial circuit through structural
  preserving iterative methods. Besides depending on a sufficiently expressive initial structure,
  these can possibly take several minutes per iteration and many iterations until visible
  improvement. Lastly, other approaches involve randomly generating a probabilistic circuit by some
  criterion. Although usually less performant compared to other methods, random PCs are orders of
  magnitude more time efficient. With this in mind, this dissertation aims to propose fast and
  scalable random structure learning algorithms for PCs from two different standpoints: from a
  logical point of view, we efficiently construct a highly structured binary PC that takes certain
  knowledge in the form of logical constraints and scalably translate them into a probabilistic
  circuit; from the viewpoint of data guided structure search, we propose hierarchically building
  PCs from random hyperplanes. We empirically show that either approach is competitive against
  state-of-the-art methods of the same class, and that their performance can be further boosted by
  simple ensemble strategies.
}

\resumo{
}
