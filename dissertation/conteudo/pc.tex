\chapter{Probabilistic Circuits}
\label{ch:pc}

As we briefly mentioned in the last chapter, Probabilistic Circuits (PCs) are conceptualized as
computational graphs under special conditions. In this chapter, \cref{sec:pc} to be more precise,
we formally define PCs and give an intuition on their syntax, viewing other probabilistic models
through the lenses of the PC framework. In \cref{sec:const}, we describe the special structural
constraints that give PCs their inference power over other generative models and state which
queries (as far as we know) are enabled from each constraint.

\section{Distributions as Computational Graphs}
\label{sec:pc}

Probabilistic circuits are directed acyclic graphs usually recursively defined in terms of their
computational units. In its simplest form a PC is a single unit with no outgoing edges whose value
corresponds to the result of a function. These are often called \emph{input} nodes, and can take
any form as long as its value is tractably computable. More concretely, input nodes typically
represent probability density (or mass) functions, although they also support inputs as joint
probability density functions of complex non-parametric models as well. To simplify notation, from
here on out we shall use the term distribution and probability density (resp. mass) function
interchangeably and argue that the input node represents a probability distribution.
\begin{example}[sidebyside,lefthand width=0.6\textwidth]{Gaussians as probabilistic circuits}{gaussians} %
  Let $\Leaf_p$ an input node and $p(X)=\gaussian(X;\mu,\sigma^2)$ a univariate Gaussian
  distribution. Computing any query on $\Leaf_p$ is straightforward: any query on
  $\Leaf_p$ directly translates to $p$. As an example, suppose $\mu=0$ and $\sigma^2=1$ and
  we wish to compute $\Leaf_p(x=0.8)$. The probability of this input shall then be
  \begin{equation*}
    \Leaf_p(x=0.8)=\gaussian(x=0.8;\mu=0,\sigma^2=1)=0.29.
  \end{equation*}
  \tcblower
  \begin{center}
    \begin{tikzpicture}
      \pgfplotsset{
        every axis/.append style={
          axis line style={->},
          tick label style={font={\scriptsize\bfseries}},
          x tick label style={color=white,above right},
          y tick label style={color=white,above right},
          grid style={black,dashed},
        }
      }
      \begin{axis}[
        no markers, domain=-5:5, samples=35,
        height=2.75cm, width=\columnwidth,
        xtick={0.8}, ytick={0.29},
        xticklabels={\colorbox{boxblue}{\textbf{0.8}}},
        yticklabels={\colorbox{boxgreen}{\textbf{0.29}}},
        axis lines*=left, xlabel=$x$, ylabel=$p(x)$,
        every axis y label/.style={font=\scriptsize,at={(axis description cs:-0.1,0.9)},anchor=south},
        every axis x label/.style={font=\scriptsize,at=(current axis.right of origin),anchor=west},
        enlargelimits=false, clip=false, axis on top,
        grid = major
      ]
        \addplot[very thick,boxteal] {gauss(0,1)};
      \end{axis}
    \end{tikzpicture}

    \begin{tikzpicture}
      \node (x) at (0, 0) {$x$};
      \newGaussNode{g}{$(x) + (1,0)$};
      \node (px) at ($(g) + (1.25,0)$) {$p(x)$};
      \draw[boxdgray,edge] (x) -- (g);
      \draw[boxdgray,edge] (g) -- (px);

      \newGaussNode[label=below:$X$]{gv}{$(g) - (0,1.0)$};
      \node (xv) at ($(gv) - (1.25,0)$) {\colorbox{boxblue}{\color{white}$\mathbf{.80}$}};
      \node (pxv) at ($(gv) + (1.25,0)$) {\colorbox{boxgreen}{\color{white}$\mathbf{.29}$}};
      \draw[boxdgray,edge] (xv) -- (gv);
      \draw[boxdgray,edge] (gv) -- (pxv);
    \end{tikzpicture}
  \end{center}
\end{example}
Let $\Leaf_p$ a PC input node and denote $p$ as its inherent probability distribution. By
definition, any query $f:\mathcal{X}\to \mathcal{Y}$ which is tractable on $p$ is tractable on
$\Leaf_p$. We shall denote $\Leaf_p(\set{X})=p(\set{X})$, and often omit $p$ when its explicit form
is not needed. Evidently, a single input node lacks the expressivity for modeling complex models, otherwise we
would have just used the input distribution as a standalone model. The expressiveness of PCs comes
from recursively combining distributions into complex functions. This can be done through
computational units that either compute convex combinations or products of their children. Let us
first look at convex combinations, known in the literature as \emph{sum} nodes.

Let $\Sum$ be a PC sum node, and denote by $\Ch(\Sum)$ the children nodes of $\Sum$. For every edge
$\edge{\Sum\Child}$ coming out of $\Sum$ and going to $\Child$, we attribute a weight $w_{\Sum,
\Child}>0$, such that $\sum_{\Child\in\Ch(\Sum)} w_{\Sum,\Child} = 1$. A sum node semantically
defines a mixture model over its children, essentially acting as a latent variable over the
component distributions \citep{poon11,peharz16}. Its value is the weighted sum of its children
$p_{\Sum}(\set{X}=\set{x})=\sum_{\Child\in\Ch(\Sum)}w_{\Sum,\Child}\cdot
p_{\Child}(\set{X}=\set{x})$. To simplify notation, we shall use $\Node(\set{X}=\set{x})$ as an
alias for $p_{\Node}(\set{X}=\set{x})$, that is, the probability function given by $\Node$'s
induced distribution.

\newcommand\mone{1}%
\newcommand\sone{0.65}%
\newcommand\mtwo{2.5}%
\newcommand\stwo{0.85}%
\newcommand\mthr{4}%
\newcommand\sthr{0.6}%
\begin{example}[sidebyside,lefthand width=0.55\textwidth]{Gaussian mixture models as probabilistic circuits}{mixgauss}
  A Gaussian Mixture Model (GMM) defines a mixture over Gaussian components. Say we wish to compute
  the probability of $X=x$ for a GMM $\mathcal{G}$ with three components
  $\gaussian_1(\mu_1=\mone,\sigma^2_1=\sone)$, $\gaussian_2(\mu_2=\mtwo, \sigma^2_1=\stwo)$ and
  $\gaussian_3(\mu_3=\mthr,\sigma^2_3=\sthr)$, and suppose we have weights set to
  $\phi=(0.4,0.25,0.35)$. Computing the probability of $G$ amounts to the weighted summation
  \begin{align*}
    \mathcal{G}(X=x)=&0.4\cdot\gaussian_1(x;\mu_1,\sigma^2_2)+0.25\cdot\gaussian_2(x;\mu_2,\sigma^2_2)+\\
                     &0.35\cdot\gaussian_3(x;\mu_3,\sigma^2_3),
  \end{align*}
  which is equivalent to a computational graph (i.e. a PC) with a sum node whose weights are set to
  $\phi$ and children are the components of the mixture. The figure on the right shows
  $\mathcal{G}$ (top) and its corresponding PC (middle). Given $x=1.5$ (in blue), input nodes are
  computed following the inference flow (bottom, gray edges) up to the root sum node (in red),
  where a weighted summation is computed to output the probability (in green).

  \tcblower
  \begin{center}
    \begin{tikzpicture}
      \pgfplotsset{
        every axis/.append style={
          axis line style={->},
          tick label style={font={\scriptsize\bfseries}},
          x tick label style={color=white,below},
          y tick label style={color=white,left},
          grid style={black,dashed},
        }
      }
      \begin{axis}[
        no markers, domain=-1:6, samples=35,
        height=3.75cm, width=\columnwidth,
        xtick={1.5}, ytick={0.2414},
        xticklabels={\colorbox{boxblue}{\textbf{1.5}}},
        yticklabels={\colorbox{boxgreen}{\textbf{0.24}}},
        axis lines*=left, xlabel=$x$, ylabel=$p(x)$,
        every axis y label/.style={font=\scriptsize,at={(axis description cs:-0.1,0.9)},anchor=south},
        every axis x label/.style={font=\scriptsize,at=(current axis.right of origin),anchor=west},
        enlargelimits=false, clip=false, axis on top,
        grid = major
      ]
        \path[name path=axis] (axis cs:0,0) -- (axis cs:5,0);
        \addplot[very thick,boxteal,name path=g1] {gauss(\mone,\sone)};
        \addplot[very thick,boxorange,name path=g2] {gauss(\mtwo,\stwo)};
        \addplot[very thick,boxpurple,name path=g3] {gauss(\mthr,\sthr)};
        \addplot[very thick,boxred] {mixgauss3(\mone,\sone,\mtwo,\stwo,\mthr,\sthr,0.4,0.25,0.35)};
        \addplot[boxteal!60] fill between [of=g1 and axis];
        \addplot[boxorange!50] fill between [of=g2 and axis];
        \addplot[boxpurple!40] fill between [of=g3 and axis];
        \node at (\mone, 0.7) {\tiny$\mu_1=\mone$};
        \node at (\mtwo, 0.5) {\tiny$\mu_2=\mtwo$};
        \node at (\mthr, 0.75) {\tiny$\mu_3=\mthr$};
      \end{axis}
    \end{tikzpicture}

    \begin{tikzpicture}
      \newSumNode[fill=boxred!70]{s}{0,0};
      \newGaussNode[fill=boxteal]{g1}{$(s) + (-1.5,-1)$};
      \newGaussNode[fill=boxorange!80]{g2}{$(s) + (0,-1)$};
      \newGaussNode[fill=boxpurple!60]{g3}{$(s) + (1.5,-1)$};
      \draw[edge] (s) edge (g1);
      \draw[edge] (s) edge (g2);
      \draw[edge] (s) edge (g3);
      \node at ($(s) + (-1.2,-0.35)$) {\scriptsize$.40$};
      \node at ($(s) + (-0.3,-0.5)$) {\scriptsize$.25$};
      \node at ($(s) + (1,-0.35)$) {\scriptsize$.35$};
      \node (l1) at ($(g1) + (0,-0.5)$) {\scriptsize$\gaussian_1(\mone,\sone)$};
      \node (l2) at ($(g2) + (0,-0.5)$) {\scriptsize$\gaussian_2(\mtwo,\stwo)$};
      \node (l3) at ($(g3) + (0,-0.5)$) {\scriptsize$\gaussian_3(\mthr,\sthr)$};
    \end{tikzpicture}

    \begin{tikzpicture}
      \newSumNode[fill=boxred!70]{s}{0,0};
      \newGaussNode[fill=boxteal]{g1}{$(s) + (-1.5,-1)$};
      \newGaussNode[fill=boxorange!80]{g2}{$(s) + (0,-1)$};
      \newGaussNode[fill=boxpurple!60]{g3}{$(s) + (1.5,-1)$};
      \draw[edge] (s) edge[bend right=5] (g1);
      \draw[edge] (s) edge[bend right=5] (g2);
      \draw[edge] (s) edge[bend right=5] (g3);
      \node at ($(s) + (-1.2,-0.35)$) {\scriptsize$.40$};
      \node at ($(s) + (-0.3,-0.5)$) {\scriptsize$.25$};
      \node at ($(s) + (1,-0.35)$) {\scriptsize$.35$};
      \node (inp) at ($(g2) + (0,-1.5)$) {\scriptsize\colorbox{boxblue}{\color{white}$\mathbf{1.5}$}};
      \node (out) at ($(s) + (0,1.0)$) {\scriptsize\colorbox{boxgreen}{\color{white}$\mathbf{0.24}$}};
      \draw[edge,boxdgray] (s) -- (out);
      \node (l1) at ($(g1) + (0,-0.5)$) {\scriptsize\colorbox{boxteal}{\color{white}$\mathbf{0.45}$}};
      \node (l2) at ($(g2) + (0,-0.5)$) {\scriptsize\colorbox{boxorange}{\color{white}$\mathbf{0.23}$}};
      \node (l3) at ($(g3) + (0,-0.5)$) {\scriptsize\colorbox{boxpurple!80}{\color{white}$\mathbf{0.00}$}};
      \draw[edge,boxdgray] (inp) edge (l1);
      \draw[edge,boxdgray] (inp) edge (l2);
      \draw[edge,boxdgray] (inp) edge (l3);
      \draw[edge,boxdgray] (g1) edge[bend left=-5] (s);
      \draw[edge,boxdgray] (g2) edge[bend left=-5] (s);
      \draw[edge,boxdgray] (g3) edge[bend left=-5] (s);
    \end{tikzpicture}
  \end{center}
\end{example}

So far, the only nonlinearities present in PCs come from the internal computations of input nodes.
In fact, a PC that only contains sums inputs can always be reduced to a sum node rooted PC with a
single layer, i.e. a mixture model (see \cref{thm:summix}). Adding \emph{product nodes} as another
form of nonlinearity increases expressivity sufficiently for PCs to be capable of representing any
(discrete) probability distribution \citep{darwiche03,martens14,peharz15}. More importantly,
products semantically act as factorizations of their children, indicating an independence
relationship between variables from different children. In practice, product nodes are simply
products of their childrens' distribution: if $\Prod$ is a PC product node, then its value is given
by $\Prod(\set{X}=\set{x})=\prod_{\Child\in\Ch(\Prod)}\Child(\set{X}=\set{x})$.

\newcommand\xmone{2}%
\newcommand\xsone{0.5}%
\newcommand\xmtwo{4}%
\newcommand\xstwo{0.8}%
\newcommand\ymone{3}%
\newcommand\ysone{0.7}%
\newcommand\ymtwo{5}%
\newcommand\ystwo{0.4}%
\begin{example}[sidebyside,lefthand width=0.55\textwidth]{Factors as probabilistic circuits}{factors}
  Say we have two GMMs $\mathcal{G}_1$ and $\mathcal{G}_2$. The first is a mixture model over
  variable $X$, with component weights $\phi_1=(0.3,0.7)$ and gaussians
  $\gaussian_1(\mu_1=\xmone,\sigma_1=\xsone)$ and $\gaussian_2(\mu_2=\xmtwo,\sigma_2=\xstwo)$. The
  second is composed of $\gaussian_3(\mu_3=\ymone,\sigma_2=\ysone)$ and $\gaussian_4(\mu_4=\ymtwo,
  \sigma_2=\ystwo)$, both distributions over variable $Y$ and with weights $\phi_2=(0.6,0.4)$.

  Suppose $X\indep Y$, yet we wish to compute the joint probability of both $x$ and $y$. If
  $X\indep Y$, then $p(x,y)=p(x)p(y)=\mathcal{G}_1(x)\mathcal{G}_2(y)$, which corresponds to a
  factoring of mixtures. This is represented as a product node (in green) over the two mixture
  models (in red and purple). The resulting joint of this circuit is shown below.
\begin{tikzpicture}
  \pgfplotsset{
    every axis/.append style={
      axis line style={->},
      axis lines=center,
      grid style={black,dashed},
      x tick label style={color=white,below},
      y tick label style={color=white,right},
      z tick label style={color=white,left},
    }
  }
  \begin{axis}[
    no markers, width=0.9\columnwidth,
    xtick={2}, ytick={4}, ztick={0.035},
    xticklabels={\colorbox{boxblue}{\textbf{2}}},
    yticklabels={\colorbox{boxblue}{\textbf{4}}},
    zticklabels={\colorbox{boxgreen}{\textbf{0.035}}},
    xlabel=$x$, ylabel=$y$, zlabel={$p(x,y)$},
    axis lines*=left,
    xlabel style={anchor=north west},
    ylabel style={anchor=south west},
    zlabel style={anchor=south west},
    enlargelimits=false, clip=false, axis on top,
    grid = major
  ]
    \addplot3[
      surf, samples=50,
      domain=0.5:6.5,
      y domain=1:6
    ] {((0.3*exp(-((x-\xmone)^2)/(2*\xsone^2))/\xsone+0.7*exp(-((x-\xmtwo)^2)/(2*\xstwo^2))/\xstwo)/2.5066)*(((0.6*exp(-((y-\ymone)^2)/(2*\ysone^2))/\ysone+0.4*exp(-((y-\ymtwo)^2)/(2*\ystwo^2))/\ystwo)/2.5066))};
  \end{axis}
\end{tikzpicture}

  \tcblower
  \begin{center}
    \begin{tikzpicture}
      \pgfplotsset{
        every axis/.append style={
          axis line style={->},
          tick label style={font={\scriptsize\bfseries}},
          x tick label style={color=white,below},
          y tick label style={color=white,left},
          grid style={black,dashed},
        }
      }
      \begin{axis}[
        no markers, domain=0:7, samples=35,
        height=3.0cm, width=\columnwidth,
        xtick={2}, ytick={0.25},
        xticklabels={\colorbox{boxblue}{\textbf{2}}},
        yticklabels={\colorbox{boxred}{\textbf{0.25}}},
        axis lines*=left, xlabel=$x$, ylabel=$p(x)$,
        every axis y label/.style={font=\scriptsize,at={(axis description cs:-0.1,0.9)},anchor=south},
        every axis x label/.style={font=\scriptsize,at=(current axis.right of origin),anchor=west},
        enlargelimits=false, clip=false, axis on top,
        grid = major
      ]
        \path[name path=axis] (axis cs:0,0) -- (axis cs:7,0);
        \addplot[very thick,boxteal,name path=g1] {gauss(\xmone,\xsone)};
        \addplot[very thick,boxorange,name path=g2] {gauss(\xmtwo,\xstwo)};
        \addplot[very thick,boxred] {mixgauss2(\xmone,\xsone,\xmtwo,\xstwo,0.3,0.7)};
        \addplot[boxteal!60] fill between [of=g1 and axis];
        \addplot[boxorange!50] fill between [of=g2 and axis];
        \node at (axis cs:\xmone,{egauss(\xmone,\xsone,\xmone)+0.1}) {\tiny$\mu_1=\xmone$};
        \node at (axis cs:\xmtwo,{egauss(\xmtwo,\xstwo,\xmtwo)+0.1}) {\tiny$\mu_2=\xmtwo$};
      \end{axis}
    \end{tikzpicture}
    \begin{tikzpicture}
      \pgfplotsset{
        every axis/.append style={
          axis line style={->},
          tick label style={font={\scriptsize\bfseries}},
          x tick label style={color=white,below},
          y tick label style={color=white,left},
          grid style={black,dashed},
        }
      }
      \begin{axis}[
        no markers, domain=1:7, samples=50,
        height=3.0cm, width=\columnwidth,
        xtick={4}, ytick={0.14},
        xticklabels={\colorbox{boxblue}{\textbf{4}}},
        yticklabels={\colorbox{boxpurple}{\textbf{0.14}}},
        axis lines*=left, xlabel=$y$, ylabel=$p(y)$,
        every axis y label/.style={font=\scriptsize,at={(axis description cs:-0.1,0.9)},anchor=south},
        every axis x label/.style={font=\scriptsize,at=(current axis.right of origin),anchor=west},
        enlargelimits=false, clip=false, axis on top,
        grid = major
      ]
        \path[name path=axis] (axis cs:0,0) -- (axis cs:7,0);
        \addplot[very thick,boxpink,name path=g1] {gauss(\ymone,\ysone)};
        \addplot[very thick,boxgoldenrod,name path=g2] {gauss(\ymtwo,\ystwo)};
        \addplot[very thick,boxpurple] {mixgauss2(\ymone,\ysone,\ymtwo,\ystwo,0.6,0.4)};
        \addplot[boxpink!40] fill between [of=g1 and axis];
        \addplot[boxgoldenrod!50] fill between [of=g2 and axis];
        \node at (axis cs:\ymone,{egauss(\ymone,\ysone,\ymone)+0.1}) {\tiny$\mu_3=\ymone$};
        \node at (axis cs:\ymtwo,{egauss(\ymtwo,\ystwo,\ymtwo)+0.1}) {\tiny$\mu_4=\ymtwo$};
      \end{axis}
    \end{tikzpicture}
    \begin{tikzpicture}
      \newProdNode[fill=boxgreen]{r}{0,0};
      \newSumNode[fill=boxred!70]{p}{$(r) + (-1.25,-0.75)$};
      \newSumNode[fill=boxpurple!60]{q}{$(r) + (1.25,-0.75)$};
      \newGaussNode[fill=boxteal]{x1}{$(p) + (-0.6,-1)$};
      \newGaussNode[fill=boxorange!80]{x2}{$(p) + (0.65,-1)$};
      \newGaussNode[fill=boxpink!50]{y1}{$(q) + (-0.6,-1)$};
      \newGaussNode[fill=boxgoldenrod!70]{y2}{$(q) + (0.6,-1)$};
      \draw[edge] (r) edge (p);
      \draw[edge] (r) edge (q);
      \draw[edge] (p) edge (x1);
      \draw[edge] (p) edge (x2);
      \draw[edge] (q) edge (y1);
      \draw[edge] (q) edge (y2);
      \node at ($(p) + (-0.5,-0.4)$) {\scriptsize$.3$};
      \node at ($(p) + (0.5,-0.4)$) {\scriptsize$.7$};
      \node at ($(q) + (-0.5,-0.4)$) {\scriptsize$.6$};
      \node at ($(q) + (0.5,-0.4)$) {\scriptsize$.4$};
      \node (l1) at ($(x1) + (0,-0.5)$) {\scriptsize$\gaussian_1(\xmone,\xsone)$};
      \node (l2) at ($(x2) + (0,-0.5)$) {\scriptsize$\gaussian_2(\xmtwo,\xstwo)$};
      \node (l3) at ($(y1) + (0,-0.5)$) {\scriptsize$\gaussian_1(\ymone,\ysone)$};
      \node (l4) at ($(y2) + (0,-0.5)$) {\scriptsize$\gaussian_2(\ymtwo,\ystwo)$};
    \end{tikzpicture}
    \begin{tikzpicture}
      \newProdNode[fill=boxgreen]{r}{0,0};
      \newSumNode[fill=boxred!70]{p}{$(r) + (-1.25,-0.75)$};
      \newSumNode[fill=boxpurple!60]{q}{$(r) + (1.25,-0.75)$};
      \newGaussNode[fill=boxteal]{x1}{$(p) + (-0.6,-1)$};
      \newGaussNode[fill=boxorange!80]{x2}{$(p) + (0.65,-1)$};
      \newGaussNode[fill=boxpink!50]{y1}{$(q) + (-0.6,-1)$};
      \newGaussNode[fill=boxgoldenrod!70]{y2}{$(q) + (0.6,-1)$};
      \draw[edge] (r) edge[bend right=5] (p);
      \draw[edge] (r) edge[bend right=5] (q);
      \draw[edge] (p) edge[bend right=5] (x1);
      \draw[edge] (p) edge[bend right=5] (x2);
      \draw[edge] (q) edge[bend right=5] (y1);
      \draw[edge] (q) edge[bend right=5] (y2);
      \node at ($(p) + (-0.5,-0.4)$) {\scriptsize$.3$};
      \node at ($(p) + (0.5,-0.4)$) {\scriptsize$.7$};
      \node at ($(q) + (-0.5,-0.4)$) {\scriptsize$.6$};
      \node at ($(q) + (0.5,-0.4)$) {\scriptsize$.4$};
      \node (l1) at ($(x1) + (0,-0.5)$) {\scriptsize\colorbox{boxteal}{\color{white}$\mathbf{0.24}$}};
      \node (l2) at ($(x2) + (0,-0.5)$) {\scriptsize\colorbox{boxorange}{\color{white}$\mathbf{0.01}$}};
      \node (l3) at ($(y1) + (0,-0.5)$) {\scriptsize\colorbox{boxpink}{\color{white}$\mathbf{0.12}$}};
      \node (l4) at ($(y2) + (0,-0.5)$) {\scriptsize\colorbox{boxgoldenrod}{\color{white}$\mathbf{0.01}$}};
      \draw[edge,boxdgray] (p) edge[bend left=-5] (r);
      \draw[edge,boxdgray] (q) edge[bend left=-5] (r);
      \draw[edge,boxdgray] (x1) edge[bend left=-5] (p);
      \draw[edge,boxdgray] (x2) edge[bend left=-5] (p);
      \draw[edge,boxdgray] (y1) edge[bend left=-5] (q);
      \draw[edge,boxdgray] (y2) edge[bend left=-5] (q);
      \node at ($(p) + (-0.3,0.6)$) {\scriptsize\colorbox{boxred}{\color{white}$\mathbf{0.25}$}};
      \node at ($(q) + (0.3,0.6)$) {\scriptsize\colorbox{boxpurple}{\color{white}$\mathbf{0.14}$}};
      \node (x) at ($(p) + (0,-2.5)$) {\scriptsize\colorbox{boxblue}{\color{white}$\mathbf{x=2}$}};
      \node (y) at ($(q) + (0,-2.5)$) {\scriptsize\colorbox{boxblue}{\color{white}$\mathbf{y=4}$}};
      \draw[edge,boxdgray] (x) edge (l1);
      \draw[edge,boxdgray] (x) edge (l2);
      \draw[edge,boxdgray] (y) edge (l3);
      \draw[edge,boxdgray] (y) edge (l4);
      \node (out) at ($(r) + (0,0.9)$) {\scriptsize\colorbox{boxgreen}{\color{white}$\mathbf{0.035}$}};
      \draw[edge,boxdgray] (r) edge (out);
    \end{tikzpicture}
  \end{center}
\end{example}

Now that we have introduced the three most important computational units in PCs, we are finally
ready to formally define probabilistic circuits.

\begin{definition}[Probabilistic circuit]
  A probabilistic circuits $\mathcal{C}$ is a rooted DAG whose nodes compute any tractable
  operation of their children, usually either convex combinations, known as \emph{sum} nodes, or
  \emph{products}. Nodes with no outgoing edges, i.e. \emph{input} nodes, are tractable nonnegative
  functions whose integrals exist and equal to one. Computing a value from $\mathcal{C}$ amounts to
  a bottom-up feedforward pass from input nodes to root.
  \label{def:pc}
\end{definition}

While we assume that \emph{tractable} operation or function is acceptable, we are usually
interested in $\bigo(1)$ time computable operations, and often assume the same of input functions
to simplify analysis. Further, in this dissertation we are only interested in convex combinations
and products, and as such only these operations are considered. When a probabilistic circuit
$\mathcal{C}$ contains no consecutive sums or products (i.e. for every sum all of its children are
either inputs or products and respectively for products) then it is said to be a \emph{standard}
form circuit. Any PC can be transformed into a \emph{standardized} circuit, a process we call
\emph{standardization} (see \cref{thm:standard}).

\begin{remark}[breakable]{On operators and tractability}{optract}
  Throughout this work we consider only products and convex combinations (apart from the implicit
  operations contained within input nodes) as potential computational units. The question of whether
  any other operator could be used to gain expressivity without loss of tractability is without a
  doubt an interesting research question, and one that is actively being pursued. However, this is
  certainly out of the scope of this dissertation, and so we restrict discussion on this topic and
  only give a brief comment on operator tractability here, pointing to existing literature in this
  area of research.

  \citet{friesen16} formalize the notion of replacing sums and products in PCs with any pair of
  operators in a commutative semiring, giving results on the conditions for marginalization to be
  tractable. They provide examples of common semirings and to which known formalisms they
  correspond to. One such example are PCs under the Boolean semiring $(\{0,1\},\vee,\wedge,0,1)$
  for logical inference, which are equivalent to Negation Normal Form (NNF, \cite{barwise82}) and
  constitute an instance of Logic Circuits (LCs), of which Sentential Decision Diagrams (SDDs,
  \cite{darwiche11}) and Binary Decision Diagrams (BDDs, \cite{akers78}) are a part of. Another
  less common semiring in PCs is the real min-sum semiring $(\mathbb{R}_{\infty}, \min,+,\infty,0)$
  for nonconvex optimization \citep{friesen15}.

  Recently, \citet{vergari21} extensively covered tractability conditions and complexity bounds for
  convex combinations, products, $\exp$ (and more generally powers in both naturals and reals),
  quotients and logarithms, even giving results for complex information-theoretic queries, such as
  entropies and divergences. Notably, they analyze whether structural constraints (and thus, in a
  sense, tractability) under these conditions are preserved.

  Up to now, we have only considered summations as nonnegative weighted sums. Indeed, in most
  literature the sum node is defined as a convex combination. However, negative weights have
  appeared in Logistic Circuits \citep{liang19} for discriminative modeling; and in Probabilistic
  Generating Circuits \citep{zhang21}, a class of tractable probabilistic models that subsume PCs.
  \citet{maua17a} extend (nonnegative) weights in sum nodes with probability intervals, effectively
  inducing a credal set \citep{cozman00} for measuring imprecision.

  Other works include PCs with quotients \citep{sharir18a}, transformations \citep{pevny20a}, max
  \citep{melibari16}, and einsum \citep{peharz20b} operations.
\end{remark}

Before we address the key components that make PCs interesting tractable
probabilistic models, we must first discuss some important concepts that often come up in PC
literature. Mainly, we are interested in defining two notions here: the scope of a unit and induced
subcircuits.

In simple terms, the scope of a computational unit $\Node$ of a PC is merely the set of all
variables that appear in the descendants of $\Node$. More formally, denote by $\Sc(\Node)$
the set of all variables that appear in $\Node$.  We inductively compute the scope of circuit by a
bottom-up approach: the scope of an input node $\Leaf_p$ is the set of variables that appear in
$p$'s distribution\footnote{Although we previously defined input nodes as mere functions, here we
are explicitly associating a random variable to a \emph{probability} function. Indeed, if we
are being rigorous, we should define input nodes as a pair of random variable and function. To save
space we instead assume, as previously stated, that the function is seen as both the probability
(density) function as well as the distribution itself, and thus its scope is the scope of its
distribution, i.e. the random variables that come into play in a probability distribution.}, and
the scope of any other node is the union of all of its childrens' scopes.  The notion of scope is
essential to the structural constraints seen in \cref{sec:const}.

As an example, take the circuit from \Cref{eg:factors}. The scope of input nodes
\inode[fill=boxteal]{\newGaussNode} and \inode[fill=boxorange!80]{\newGaussNode} are
$\Sc(\inode[fill=boxteal]{\newGaussNode})=\Sc(\inode[fill=boxorange!80]{\newGaussNode})=\{X\}$,
while $\Sc(\inode[fill=boxpink!50]{\newGaussNode})=\Sc(\inode[fill=boxgoldenrod!70]{\newGaussNode})
=\{Y\}$. Consequentially, their parent sum nodes will have the same scope as their children
$\Sc(\inode[fill=boxred!70]{\newSumNode})=\{X\}$ and $\Sc(\inode[fill=boxpurple!60]{\newSumNode})=
\{Y\}$, yet the root node's scope is $\Sc(\inode[fill=boxgreen]{\newProdNode})=\{X,Y\}$, since its
childrens' scopes are distinct.

Let $\mathcal{C}$ a probabilistic circuit and node $\Node\in\mathcal{C}$. We say that
$\mathcal{S}_{\Node}$ is a subcircuit of $\mathcal{C}$ rooted at $\Node$ if $\mathcal{S}_{\Node}$'s
root is $\Node$, all nodes and edges in $\mathcal{S}_{\Node}$ are also in $\mathcal{C}$ and
$\mathcal{S}$ is also a probabilistic circuit. We now introduce the concept of induced subcircuits
\citep{chan06,dennis15,peharz14}.

\begin{definition}[Induced subcircuit]
  Let $\mathcal{C}$ a probabilistic circuit. An induced subcircuit $\mathcal{S}$ of $\mathcal{C}$
  is a subcircuit of $\mathcal{C}$ such that all edges coming out of product nodes in $\mathcal{C}$
  are also in $\mathcal{S}$, and of all edges coming out of sum nodes in $\mathcal{C}$, only one is
  in $\mathcal{S}$.
\end{definition}

\begin{figure}[t]
  \begin{subfigure}[t]{0.245\textwidth}
    \begin{center}
      \resizebox{\textwidth}{!}{
      \begin{tikzpicture}
        \newSumNode[fill=boxgreen]{r}{0,0};
        \newProdNode[fill=boxred!70]{p1}{$(r) + (-1.5,-1)$};
        \newProdNode[fill=boxred!70]{p2}{$(r) + (0,-1)$};
        \newProdNode[fill=boxred!70]{p3}{$(r) + (1.5,-1)$};
        \newSumNode[fill=boxpurple!60]{s1}{$(p2) + (-2.0,-1)$};
        \newSumNode[fill=boxpurple!60]{s2}{$(p2) + (-0.66,-1)$};
        \newSumNode[fill=boxpurple!60]{s3}{$(p2) + (0.66,-1)$};
        \newSumNode[fill=boxpurple!60]{s4}{$(p2) + (2.0,-1)$};
        \newGaussNode[fill=boxteal]{g1}{$(s1) + (0,-1)$};
        \newGaussNode[fill=boxorange!80]{g2}{$(s2) + (0,-1)$};
        \newGaussNode[fill=boxpink!50]{g3}{$(s3) + (0,-1)$};
        \newGaussNode[fill=boxgoldenrod!70]{g4}{$(s4) + (0,-1)$};
        \draw[edge] (r) edge (p1);
        \draw[edge] (r) edge (p2);
        \draw[edge] (r) edge (p3);
        \draw[edge] (p1) edge (s1);
        \draw[edge] (p1) edge (s3);
        \draw[edge] (p2) edge (s2);
        \draw[edge] (p2) edge (s3);
        \draw[edge] (p3) edge (s2);
        \draw[edge] (p3) edge (s4);
        \draw[edge] (s1) edge (g1);
        \draw[edge] (s1) edge (g2);
        \draw[edge] (s2) edge (g1);
        \draw[edge] (s2) edge (g2);
        \draw[edge] (s3) edge (g3);
        \draw[edge] (s3) edge (g4);
        \draw[edge] (s4) edge (g3);
        \draw[edge] (s4) edge (g4);
      \end{tikzpicture}
      }
    \end{center}
    \caption{}
    \label{fig:pc}
  \end{subfigure}
  \begin{subfigure}[t]{0.245\textwidth}
    \begin{center}
      \resizebox{\textwidth}{!}{
      \begin{tikzpicture}
        \newSumNode[fill=boxgreen]{r}{0,0};
        \newProdNode[fill=boxred!70]{p1}{$(r) + (-1.5,-1)$};
        \newProdNode[fill=boxred!70]{p2}{$(r) + (0,-1)$};
        \newProdNode[fill=boxred!70]{p3}{$(r) + (1.5,-1)$};
        \newSumNode[fill=boxpurple!60]{s1}{$(p2) + (-2.0,-1)$};
        \newSumNode[fill=boxpurple!60]{s2}{$(p2) + (-0.66,-1)$};
        \newSumNode[fill=boxpurple!60]{s3}{$(p2) + (0.66,-1)$};
        \newSumNode[fill=boxpurple!60]{s4}{$(p2) + (2.0,-1)$};
        \newGaussNode[fill=boxteal]{g1}{$(s1) + (0,-1)$};
        \newGaussNode[fill=boxorange!80]{g2}{$(s2) + (0,-1)$};
        \newGaussNode[fill=boxpink!50]{g3}{$(s3) + (0,-1)$};
        \newGaussNode[fill=boxgoldenrod!70]{g4}{$(s4) + (0,-1)$};
        \draw[thick,red,edge] (r) edge (p1);
        \draw[boxgray,edge] (r) edge (p2);
        \draw[boxgray,edge] (r) edge (p3);
        \draw[thick,red,edge] (p1) edge (s1);
        \draw[thick,red,edge] (p1) edge (s3);
        \draw[boxgray,edge] (p2) edge (s2);
        \draw[boxgray,edge] (p2) edge (s3);
        \draw[boxgray,edge] (p3) edge (s2);
        \draw[boxgray,edge] (p3) edge (s4);
        \draw[thick,red,edge] (s1) edge (g1);
        \draw[boxgray,edge] (s1) edge (g2);
        \draw[boxgray,edge] (s2) edge (g1);
        \draw[boxgray,edge] (s2) edge (g2);
        \draw[thick,red,edge] (s3) edge (g3);
        \draw[boxgray,edge] (s3) edge (g4);
        \draw[boxgray,edge] (s4) edge (g3);
        \draw[boxgray,edge] (s4) edge (g4);
      \end{tikzpicture}
      }
    \end{center}
  \end{subfigure}\begin{subfigure}[t]{0.245\textwidth}
    \begin{center}
      \resizebox{\textwidth}{!}{
      \begin{tikzpicture}
        \newSumNode[fill=boxgreen]{r}{0,0};
        \newProdNode[fill=boxred!70]{p1}{$(r) + (-1.5,-1)$};
        \newProdNode[fill=boxred!70]{p2}{$(r) + (0,-1)$};
        \newProdNode[fill=boxred!70]{p3}{$(r) + (1.5,-1)$};
        \newSumNode[fill=boxpurple!60]{s1}{$(p2) + (-2.0,-1)$};
        \newSumNode[fill=boxpurple!60]{s2}{$(p2) + (-0.66,-1)$};
        \newSumNode[fill=boxpurple!60]{s3}{$(p2) + (0.66,-1)$};
        \newSumNode[fill=boxpurple!60]{s4}{$(p2) + (2.0,-1)$};
        \newGaussNode[fill=boxteal]{g1}{$(s1) + (0,-1)$};
        \newGaussNode[fill=boxorange!80]{g2}{$(s2) + (0,-1)$};
        \newGaussNode[fill=boxpink!50]{g3}{$(s3) + (0,-1)$};
        \newGaussNode[fill=boxgoldenrod!70]{g4}{$(s4) + (0,-1)$};
        \draw[boxgray,edge] (r) edge (p1);
        \draw[thick,blue,edge] (r) edge (p2);
        \draw[boxgray,edge] (r) edge (p3);
        \draw[boxgray,edge] (p1) edge (s1);
        \draw[boxgray,edge] (p1) edge (s3);
        \draw[thick,blue,edge] (p2) edge (s2);
        \draw[thick,blue,edge] (p2) edge (s3);
        \draw[boxgray,edge] (p3) edge (s2);
        \draw[boxgray,edge] (p3) edge (s4);
        \draw[boxgray,edge] (s1) edge (g1);
        \draw[boxgray,edge] (s1) edge (g2);
        \draw[boxgray,edge] (s2) edge (g1);
        \draw[thick,blue,edge] (s2) edge (g2);
        \draw[boxgray,edge] (s3) edge (g3);
        \draw[thick,blue,edge] (s3) edge (g4);
        \draw[boxgray,edge] (s4) edge (g3);
        \draw[boxgray,edge] (s4) edge (g4);
      \end{tikzpicture}
      }
    \end{center}
    \caption{}
    \label{fig:subcircs}
  \end{subfigure}\begin{subfigure}[t]{0.245\textwidth}
    \begin{center}
      \resizebox{\textwidth}{!}{
      \begin{tikzpicture}
        \newSumNode[fill=boxgreen]{r}{0,0};
        \newProdNode[fill=boxred!70]{p1}{$(r) + (-1.5,-1)$};
        \newProdNode[fill=boxred!70]{p2}{$(r) + (0,-1)$};
        \newProdNode[fill=boxred!70]{p3}{$(r) + (1.5,-1)$};
        \newSumNode[fill=boxpurple!60]{s1}{$(p2) + (-2.0,-1)$};
        \newSumNode[fill=boxpurple!60]{s2}{$(p2) + (-0.66,-1)$};
        \newSumNode[fill=boxpurple!60]{s3}{$(p2) + (0.66,-1)$};
        \newSumNode[fill=boxpurple!60]{s4}{$(p2) + (2.0,-1)$};
        \newGaussNode[fill=boxteal]{g1}{$(s1) + (0,-1)$};
        \newGaussNode[fill=boxorange!80]{g2}{$(s2) + (0,-1)$};
        \newGaussNode[fill=boxpink!50]{g3}{$(s3) + (0,-1)$};
        \newGaussNode[fill=boxgoldenrod!70]{g4}{$(s4) + (0,-1)$};
        \draw[boxgray,edge] (r) edge (p1);
        \draw[boxgray,edge] (r) edge (p2);
        \draw[thick,green!20!black,edge] (r) edge (p3);
        \draw[boxgray,edge] (p1) edge (s1);
        \draw[boxgray,edge] (p1) edge (s3);
        \draw[boxgray,edge] (p2) edge (s2);
        \draw[boxgray,edge] (p2) edge (s3);
        \draw[thick,green!20!black,edge] (p3) edge (s2);
        \draw[thick,green!20!black,edge] (p3) edge (s4);
        \draw[boxgray,edge] (s1) edge (g1);
        \draw[boxgray,edge] (s1) edge (g2);
        \draw[thick,green!20!black,edge] (s2) edge (g1);
        \draw[boxgray,edge] (s2) edge (g2);
        \draw[boxgray,edge] (s3) edge (g3);
        \draw[boxgray,edge] (s3) edge (g4);
        \draw[boxgray,edge] (s4) edge (g3);
        \draw[thick,green!20!black,edge] (s4) edge (g4);
      \end{tikzpicture}
      }
    \end{center}
  \end{subfigure}
  \caption{A probabilistic circuit (a) and 3 of the 12 possible induced subcircuits (b).}
  \label{fig:induced}
\end{figure}

Examples of induced subcircuits are visualized in \cref{fig:induced}. When the induced subcircuit
is a tree, as is the case in \cref{fig:induced}, they are referred to as induced tree
\citep{zhao15,zhao16b}.

So far, by \cref{def:pc} a PC does not yet necessarily represent a probability distribution. To do
so, the structure must obey constraints that we have only previously mentioned in passing. We
formally define them next.

\section{Deciding What to Constraint}
\label{sec:const}

In this section we are interested in studying how the structural constraints in PCs enable
different inference tasks. We shall first cover the more basic queries, namely \emph{probability of
evidence} (\evi), \emph{marginal probability} (\mar), \emph{conditional probability} (\con) and
\emph{maximum a posteriori probability} (\map). After that we address more complex queries such as
\emph{mutual information} (\mi) and \emph{expectation} (\expc).

The most basic inference task we are interested in computing is the probability of evidence. To
unlock this, we must introduce two structural constraints known as \emph{smoothness} and
\emph{decomposability}.

\begin{definition}[Smoothness]
  A probabilistic circuit $\mathcal{C}$ is said to be \emph{smooth} if for every sum node $\Sum$ in
  $\mathcal{C}$, $\Sc(\Child_1)=\Sc(\Child_2)$ for $\Child_1,\Child_2\in\Ch(\Sum)$.
\end{definition}

\begin{definition}[Decomposability]
  A probabilistic circuit $\mathcal{C}$ is said to be \emph{decomposable} if for every product node
  $\Prod$ in $\mathcal{C}$, $\Sc(\Child_1)\cap\Sc(\Child_2)=\emptyset$ for $\Child_1,\Child_2\in
  \Ch(\Prod)$.
\end{definition}

For any \emph{smooth} and \emph{decomposable} PC, computing \evi{} is done in linear time in the
number of edges. In fact this is true for \mar{} and \con{} as well. To compute marginals, it is
sufficient to compute the corresponding marginals with respect to each input node and proceed to
propagate values bottom-up. For conditionals, we simply compute two passes: one where we
marginalize the conditional variables and the other any other variables that are not present in our
query. These procedures are formalized in the theorem below.

\begin{restatable}{theorem}{linevi}
  \label{thm:linevi}
  Let $\mathcal{C}$ a \emph{smooth} and \emph{decomposable} PC. Any one of \evi{}, \mar{} or \con{}
  can be computed in linear time (in the number of edges of $\mathcal{C}$).
\end{restatable}

\begin{figure}[t]
  \begin{subfigure}[t]{0.31\textwidth}
    \begin{center}
      \begin{tikzpicture}
        \newSumNode[fill=boxgreen]{r}{0,0};
        \newProdNode[fill=boxred]{p1}{$(r) + (-1.5,-1)$};
        \newProdNode[fill=boxred]{p2}{$(r) + (0,-1)$};
        \newProdNode[fill=boxred]{p3}{$(r) + (1.5,-1)$};
        \newGaussNode[fill=boxteal,label=below:{$A$}]{a}{$(p1) + (0,-1)$};
        \newGaussNode[fill=boxorange!80,label=below:{$B$}]{b}{$(p2) + (0,-1)$};
        \newGaussNode[fill=boxpink!50,label=below:{$C$}]{c}{$(p3) + (0,-1)$};
        \draw[edge] (r) edge (p1);
        \draw[edge] (r) edge (p2);
        \draw[edge] (r) edge (p3);
        \draw[edge] (p1) edge (a);
        \draw[edge] (p1) edge (b);
        \draw[edge] (p2) edge (a);
        \draw[edge] (p2) edge (c);
        \draw[edge] (p3) edge (b);
        \draw[edge] (p3) edge (c);
      \end{tikzpicture}
    \end{center}
    \caption{}
  \end{subfigure}
  \begin{subfigure}[t]{0.31\textwidth}
    \begin{center}
      \begin{tikzpicture}
        \newProdNode[fill=boxgreen]{r}{0,0};
        \newSumNode[fill=boxred]{p1}{$(r) + (-2.0,-1)$};
        \newSumNode[fill=boxred]{p2}{$(r) + (-0.66,-1)$};
        \newSumNode[fill=boxred]{p3}{$(r) + (0.66,-1)$};
        \newSumNode[fill=boxred]{p4}{$(r) + (2.0,-1)$};
        \newGaussNode[fill=boxteal,label=below:{$A$}]{a1}{$(p1) + (0,-1)$};
        \newGaussNode[fill=boxteal,label=below:{$A$}]{a2}{$(p2) + (0,-1)$};
        \newGaussNode[fill=boxorange!80,label=below:{$B$}]{b1}{$(p3) + (0,-1)$};
        \newGaussNode[fill=boxorange!80,label=below:{$B$}]{b2}{$(p4) + (0,-1)$};
        \draw[edge] (r) -- (p1);
        \draw[edge] (r) -- (p2);
        \draw[edge] (r) -- (p3);
        \draw[edge] (r) -- (p4);
        \draw[edge] (p1) -- (a1);
        \draw[edge] (p1) -- (a2);
        \draw[edge] (p2) -- (a1);
        \draw[edge] (p2) -- (a2);
        \draw[edge] (p3) -- (b1);
        \draw[edge] (p3) -- (b2);
        \draw[edge] (p4) -- (b1);
        \draw[edge] (p4) -- (b2);
      \end{tikzpicture}
    \end{center}
    \caption{}
  \end{subfigure}
  \begin{subfigure}[t]{0.31\textwidth}
    \begin{center}
      \begin{tikzpicture}
        \newProdNode[fill=boxgreen]{r}{0,0};
        \newSumNode[fill=boxred]{s1}{$(r) + (-1.0,-0.5)$};
        \newSumNode[fill=boxred]{s2}{$(r) + (1.0,-0.5)$};
        \newProdNode[fill=boxpurple!60]{p1}{$(s1) + (-0.5,-0.75)$};
        \newProdNode[fill=boxpurple!60]{p2}{$(s1) + (0.5,-0.75)$};
        \newProdNode[fill=boxpurple!60]{p3}{$(s2) + (-0.5,-0.75)$};
        \newProdNode[fill=boxpurple!60]{p4}{$(s2) + (0.5,-0.75)$};
        \newGaussNode[fill=boxteal,label=below:{$A$}]{a}{$(p1) + (0,-0.75)$};
        \newGaussNode[fill=boxorange!80,label=below:{$B$}]{b}{$(p2) + (0,-0.75)$};
        \newGaussNode[fill=boxpink!50,label=below:{$C$}]{c}{$(p3) + (0,-0.75)$};
        \newGaussNode[fill=boxgoldenrod!70,label=below:{$D$}]{d}{$(p4) + (0,-0.75)$};
        \draw[edge] (r) -- (s1);
        \draw[edge] (r) -- (s2);
        \draw[edge] (s1) -- (p1);
        \draw[edge] (s1) -- (p2);
        \draw[edge] (s2) -- (p3);
        \draw[edge] (s2) -- (p4);
        \draw[edge] (p1) -- (a);
        \draw[edge] (p1) -- (b);
        \draw[edge] (p2) -- (a);
        \draw[edge] (p2) -- (b);
        \draw[edge] (p3) -- (c);
        \draw[edge] (p3) -- (d);
        \draw[edge] (p4) -- (c);
        \draw[edge] (p4) -- (d);
      \end{tikzpicture}
    \end{center}
    \caption{}
  \end{subfigure}
  \caption{Decomposable but unsmooth (a), smooth but undecomposable (b), and smooth and
  decomposable (c) circuits.}
\end{figure}

Because \evi{}, \mar{} and \con{} are the most basic forms of querying in a PC, we shall refer to
these as the \emph{base queries}. Although smoothness and decomposability are sufficient conditions
for tractable base queries, they are not necessary conditions. As a matter of fact, decomposability
can be replaced with a third weaker constraint known as \emph{consistency}. Denote by
$\Desc(\Node)$ the set of all descendants of $\Node$.

\begin{definition}[Consistency]
  A probabilistic circuit $\mathcal{C}$ is said to be \emph{consistent} if for any product node
  $\Prod$ in $\mathcal{C}$, it holds that every two children $\Child_1,\Child_2\in\Ch(\Prod)$ there
  does not exist two input nodes $\Leaf_p^1\in\Desc(\Child_1)$ and $\Leaf_q^2\in\Desc(\Child_2)$
  whose scope is the same and $p(\set{x})\neq q(\set{x})$ for any $\set{x}\in\sspace{X}$.
\end{definition}

In \citet{peharz15}, \citeauthor{peharz15} show that although smooth and consistent PCs are more
succinct \citep{darwiche02} compared to smooth and decomposable circuits, the gain is only mild,
further proving that any smooth and consistent PC can be polynomially translated to a decomposable
equivalent. In practice, because constructing decomposable circuits (and verifying decomposability)
is much easier compared to the same with consistency, we instead focus on smooth and decomposable
PCs.

Suppose we wish to compute the most probable assignment of a variable, say for classification or
image reconstruction. To do so, we must compute the conditional query
\begin{equation*}
  \max_{\set{y}}p(\set{y}|\set{x})=\max_{\set{y}}\frac{p(\set{y},\set{x})}{p(\set{x})}=
  \frac{\max_{\set{y}}p(\set{y},\set{x})}{p(\set{x})},
\end{equation*}
often called the \emph{maximum a posteriori} (\map) probability. Although it seems at first like
\map{} is no harder than computing a \con{}, it turns out that for smooth and decomposable PCs,
\map{} is unfortunately intractable \citep{conaty17,mei18}. To unlock access to \map{} we must make
the circuit \emph{deterministic}.

\begin{definition}[Determinism]
  A probabilistic circuit $\mathcal{C}$ is said to be \emph{deterministic} if for every sum node
  $\Sum\in\mathcal{C}$ only one child of $\Sum$ has nonnegative value at a time for any assignment.
\end{definition}

At this point, we must introduce a graphical notation for \emph{indicator} nodes. An indicator node
is an input node whose function is the characteristic function $f(x)=\ind_{\{x=k\}}$, i.e. a
degenerate function with all on $k$ and zero anywhere else. A special case is when $X$ is binary
and $k=1$, in which case we say the input node is a literal, denoting by the usual propositional
notation $X$ for when $k=1$ and $\neg X$ for $k=0$. Graphically, we shall use \inode{\newLeafNode}
for indicators and the propositional notation for literals.

With determinism, we now have access to \map{}.

\begin{restatable}{theorem}{det}
  \label{thm:det}
  Let $\mathcal{C}$ a smooth, decomposable and \emph{deterministic} PC. \map{} is computable in
  linear time (in the number of edges of $\mathcal{C}$).
\end{restatable}

\begin{example}[breakable,sidebyside,lefthand width=0.55\textwidth]{Naïve Bayes as probabilistic circuits}{nbayes}
  Suppose we have census measurements on the age $A$, body mass $B$ and average amount of cheese
  eaten in a day $C$ from three different cities $Y$.  Assuming $A$, $B$ and $C$ are independent,
  given a sample $x=(a,b,c)$ we can use a Gaussian naïve Bayes to predict $x$'s class
  \begin{equation}
    p(y|a,b,c)=p(y)p(a|y)p(b|y)p(c|y).
  \end{equation}
  In PC terms, $p(y)$ are the weights for each class and $p(z|y)$ are Gaussian input nodes. To make
  sure that these are the distributions of feature $z$ given $y$, we introduce indicator variables
  ``selecting'' $Y$'s state. Since the PC is deterministic, we can compute the \map{} in linear
  time by simply replacing the root node with a max, which is equivalent to finding the highest
  likelihood of $x$ for each city $y$.
  \tcblower
  \begin{center}
  \begin{tikzpicture}
    \node[inner sep=0pt,thick,minimum size=12pt,draw,circle,fill=boxteal] (y) at (0,0) {$Y$};
    \node[inner sep=0pt,thick,minimum size=12pt,draw,circle,fill=boxorange!80] (a) at ($(y) + (-1,-1)$) {$A$};
    \node[inner sep=0pt,thick,minimum size=12pt,draw,circle,fill=boxpink!50] (b) at ($(y) + (0,-1)$) {$B$};
    \node[inner sep=0pt,thick,minimum size=12pt,draw,circle,fill=boxgoldenrod!70] (c) at ($(y) + (1,-1)$) {$C$};
    \draw[edge] (y) -- (a);
    \draw[edge] (y) -- (b);
    \draw[edge] (y) -- (c);
  \end{tikzpicture}

  \resizebox{\columnwidth}{!}{
  \begin{tikzpicture}
    \newSumNode[fill=boxgreen]{r}{0,0};
    \newProdNode[fill=boxred!70]{p1}{$(r) + (-2.25,-1)$};
    \newProdNode[fill=boxred!70]{p2}{$(r) + (0.0,-1)$};
    \newProdNode[fill=boxred!70]{p3}{$(r) + (2.25,-1)$};
    \newLeafNode[fill=boxteal,label={[label distance=-0.3cm]below left:{\scriptsize$\ind_{y=1}$}}]{i1}{$(p1) + (-1.5,-0.5)$};
    \newLeafNode[fill=boxteal,label={[label distance=-0.3cm]below left:{\scriptsize$\ind_{y=2}$}}]{i2}{$(p2) + (-1.5,-0.5)$};
    \newLeafNode[fill=boxteal,label={[label distance=-0.3cm]below left:{\scriptsize$\ind_{y=3}$}}]{i3}{$(p3) + (-1.5,-0.5)$};
    \newGaussNode[label={[label distance=-0.1cm]below left:{$A$}},fill=boxorange!80]{g11}{$(p1) + (-1.0,-1)$};
    \newGaussNode[label={[label distance=-0.1cm]below left:{$B$}},fill=boxpink!50]{g12}{$(p1) + (-0.5,-1.5)$};
    \newGaussNode[label={[label distance=-0.1cm]below left:{$C$}},fill=boxgoldenrod!70]{g13}{$(p1) + (0,-2)$};
    \newGaussNode[label={[label distance=-0.1cm]below left:{$A$}},fill=boxorange!80]{g21}{$(p2) + (-1.0,-1)$};
    \newGaussNode[label={[label distance=-0.1cm]below left:{$B$}},fill=boxpink!50]{g22}{$(p2) + (-0.5,-1.5)$};
    \newGaussNode[label={[label distance=-0.1cm]below left:{$C$}},fill=boxgoldenrod!70]{g23}{$(p2) + (0,-2)$};
    \newGaussNode[label={[label distance=-0.1cm]below left:{$A$}},fill=boxorange!80]{g31}{$(p3) + (-1.0,-1)$};
    \newGaussNode[label={[label distance=-0.1cm]below left:{$B$}},fill=boxpink!50]{g32}{$(p3) + (-0.5,-1.5)$};
    \newGaussNode[label={[label distance=-0.1cm]below left:{$A$}},fill=boxgoldenrod!70]{g33}{$(p3) + (0,-2)$};
    \draw[edge] (r) edge (p1);
    \draw[edge] (r) edge (p2);
    \draw[edge] (r) edge (p3);
    \draw[edge] (p1) edge (i1);
    \draw[edge] (p2) edge (i2);
    \draw[edge] (p3) edge (i3);
    \draw[edge] (p1) -- (g11);
    \draw[edge] (p1) -- (g12);
    \draw[edge] (p1) -- (g13);
    \draw[edge] (p2) -- (g21);
    \draw[edge] (p2) -- (g22);
    \draw[edge] (p2) -- (g23);
    \draw[edge] (p3) -- (g31);
    \draw[edge] (p3) -- (g32);
    \draw[edge] (p3) -- (g33);
  \end{tikzpicture}
  }
  \end{center}
\end{example}
