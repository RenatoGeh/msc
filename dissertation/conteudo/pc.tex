\chapter{Probabilistic Circuits}
\label{ch:pc}

As we briefly mentioned in the last chapter, Probabilistic Circuits (PCs) are conceptualized as
computational graphs under special conditions. In this chapter, \cref{sec:pc} to be more precise,
we formally define PCs and give an intuition on their syntax, viewing other probabilistic models
through the lenses of the PC framework. In \cref{sec:const}, we describe the special structural
constraints that give PCs their inference power over other generative models and state which
queries (as far as we know) are enabled from each constraint.

\section{Distributions as Computational Graphs}
\label{sec:pc}

Probabilistic circuits are directed acyclic graphs usually recursively defined in terms of their
computational units. In its simplest form, a PC is a single unit corresponding to some tractable
distribution. These can range from univariate or multivariate exponential distributions to complex
non-parametric models, and are the foundational units of PCs. We shall refer to these computational
units as \emph{leaf nodes}, as no edges come out of them. To simplify notation, in this
dissertation, unless specified, leaf nodes will denote univariate distributions. Let
$\mathcal{L}_p$ a PC leaf node and denote $p$ as its inherent probability distribution. By
definition, any query $f:\mathcal{X}\to\mathcal{Y}$ which is tractable on $p$ is tractable on
$\mathcal{L}_p$. We shall denote $\mathcal{L}_p(\set{X})=p(\set{X})$, and often omit $p$ when its
explicit form is not needed.

\begin{example}[sidebyside,lefthand width=0.6\textwidth]{Gaussians as probabilistic circuits}{gaussians}
  Let $\mathcal{L}_p$ a leaf node and $p(X)=\gaussian(X;\mu,\sigma^2)$ a univariate Gaussian
  distribution. Computing any query on $\mathcal{L}_p$ is straightforward: any query on
  $\mathcal{L}_p$ directly translates to $p$. As an example, suppose $\mu=0$ and $\sigma^2=1$ and
  we wish to compute $\mathcal{L}_p(x=0.8)$. The probability of this leaf shall then be
  \begin{equation*}
    \mathcal{L}_p(x=0.8)=\gaussian(x=0.8;\mu=0,\sigma^2=1)=0.29.
  \end{equation*}
  \tcblower
  \begin{center}
    \begin{tikzpicture}[edge/.style = {->,>=latex'}]
      \node (x) at (0, 0) {$x$};
      \newGaussNode{g}{$(x) + (1,0)$};
      \node (px) at ($(g) + (1.25,0)$) {$p(x)$};
      \draw[edge] (x) -- (g);
      \draw[edge] (g) -- (px);

      \newGaussNode[label=below:$X$]{gv}{$(g) - (0,1.0)$};
      \node (xv) at ($(gv) - (1.25,0)$) {\colorbox{boxblue}{\color{white}$\mathbf{.80}$}};
      \node (pxv) at ($(gv) + (1.25,0)$) {\colorbox{boxgreen}{\color{white}$\mathbf{.29}$}};
      \draw[edge] (xv) -- (gv);
      \draw[edge] (gv) -- (pxv);
    \end{tikzpicture}

    \begin{tikzpicture}
      \pgfplotsset{
        every axis/.append style={
          axis line style={->},
          tick label style={font={\scriptsize\bfseries}},
          x tick label style={color=white,above right},
          y tick label style={color=white,above right},
          grid style={black,dashed},
        }
      }
      \begin{axis}[
        no markers, domain=-5:5, samples=35,
        height=2.75cm, width=\columnwidth,
        xtick={0.8}, ytick={0.29},
        xticklabels={\colorbox{boxblue}{\textbf{0.8}}},
        yticklabels={\colorbox{boxgreen}{\textbf{0.29}}},
        axis lines*=left, xlabel=$x$, ylabel=$p(x)$,
        every axis y label/.style={font=\scriptsize,at={(axis description cs:-0.1,0.9)},anchor=south},
        every axis x label/.style={font=\scriptsize,at=(current axis.right of origin),anchor=west},
        enlargelimits=false, clip=false, axis on top,
        grid = major
      ]
        \addplot[very thick,boxteal] {gauss(0,1)};
      \end{axis}
    \end{tikzpicture}
  \end{center}
\end{example}

Evidently, a single leaf node lacks the expressivity for modeling complex models, otherwise we
would have just used the leaf distribution as a standalone model. The expressiveness of PCs comes
from recursively combining distributions into complex functions. This can be done through
computational units that either compute convex combinations or products of their children. Let us
first look at convex combinations, known in the literature as \emph{sum nodes}.

\begin{remark}[breakable]{On operators and tractability}{optract}
  Throughout this work we consider only products and convex combinations (apart from the implicit
  operations contained within leaf nodes) as potential computational units. The question of whether
  any other operator could be used to gain expressivity without loss of tractability is without a
  doubt an interesting research question, and one that is actively being pursued. However, this is
  certainly out of the scope of this dissertation, and so we restrict discussion on this topic and
  only give a brief comment on tractability here, pointing to existing literature in this area of
  research.

  \citet{friesen16} formalize the notion of operator generalization in PCs by looking at any
  commutative semiring, giving results on the conditions for marginalization to be tractable. They
  provide examples of common semirings and to which known formalisms they correspond to. Two such
  examples are PCs under the Boolean semiring $(\{0,1\},\vee,\wedge,0,1)$ for logical inference,
  which are equivalent to Negation Normal Form (NNF, \cite{barwise82}) and constitute an instance
  of Logic Circuits (LCs), of which Sentential Decision Diagrams (SDDs, \cite{darwiche11}) and
  Binary Decision Diagrams (BDDs, \cite{akers78}) are a part of; and PCs under the real min-sum
  semiring $(\mathbb{R}_{\infty}, \min,+,\infty,0)$ for nonconvex optimization \citep{friesen15}.

  Recently, \citet{vergari21} extensively covered tractability conditions and complexity bounds for
  convex combinations, products, $\exp$ (and more generally powers in both naturals and reals),
  divisions and logarithms, even giving results for complex information-theoretic queries, such as
  entropies and divergences.

  Up to now, we have only considered summations with nonnegative weights. Indeed, in most
  literature the sum node is defined as a convex combination. However, negative weights have
  appeared in Logistic Circuits \citep{liang19} for discriminative modeling; and in Probabilistic
  Generating Circuits \citep{zhang21}, a superset of tractable probabilistic models that subsume
  PCs.

  Other works on 
\end{remark}


Note how these second nonlinearity, the first being leaf nodes

\section{Deciding What to Constraint}
\label{sec:const}


referenced structural constraints in passing

