\chapter{Probabilistic Circuits}
\label{ch:pc}

As we briefly mentioned in the last chapter, Probabilistic Circuits (PCs) are conceptualized as
computational graphs under special conditions. In this chapter, \cref{sec:pc} to be more precise,
we formally define PCs and give an intuition on their syntax, viewing other probabilistic models
through the lenses of the PC framework. In \cref{sec:const}, we describe the special structural
constraints that give PCs their inference power over other generative models and state which
queries (as far as we know) are enabled from each constraint.

\section{Distributions as Computational Graphs}
\label{sec:pc}

Probabilistic circuits are directed acyclic graphs usually recursively defined in terms of their
computational units. In its simplest form a PC is a single unit with no outgoing edges whose value
corresponds to the result of a function. These are often called \emph{input} nodes, and can take
any form as long as its value is tractably computable. More concretely, input nodes typically
represent probability density (or mass) functions, although they also support inputs as joint
probability density functions of complex non-parametric models as well. To simplify notation, from
here on out we shall use the term distribution and probability density (resp. mass) function
interchangeably and argue that the input node represents a probability distribution.
\begin{example}[sidebyside,lefthand width=0.6\textwidth]{Gaussians as probabilistic circuits}{gaussians} %
  Let $\Leaf_p$ an input node and $p(X)=\gaussian(X;\mu,\sigma^2)$ a univariate Gaussian
  distribution. Computing any query on $\Leaf_p$ is straightforward: any query on
  $\Leaf_p$ directly translates to $p$. As an example, suppose $\mu=0$ and $\sigma^2=1$ and
  we wish to compute $\Leaf_p(x=0.8)$. The probability of this input shall then be
  \begin{equation*}
    \Leaf_p(x=0.8)=\gaussian(x=0.8;\mu=0,\sigma^2=1)=0.29.
  \end{equation*}
  \tcblower
  \begin{center}
    \begin{tikzpicture}
      \pgfplotsset{
        every axis/.append style={
          axis line style={->},
          tick label style={font={\scriptsize\bfseries}},
          x tick label style={color=white,above right},
          y tick label style={color=white,above right},
          grid style={black,dashed},
        }
      }
      \begin{axis}[
        no markers, domain=-5:5, samples=35,
        height=2.75cm, width=\columnwidth,
        xtick={0.8}, ytick={0.29},
        xticklabels={\colorbox{boxblue}{\textbf{0.8}}},
        yticklabels={\colorbox{boxgreen}{\textbf{0.29}}},
        axis lines*=left, xlabel=$x$, ylabel=$p(x)$,
        every axis y label/.style={font=\scriptsize,at={(axis description cs:-0.1,0.9)},anchor=south},
        every axis x label/.style={font=\scriptsize,at=(current axis.right of origin),anchor=west},
        enlargelimits=false, clip=false, axis on top,
        grid = major
      ]
        \addplot[very thick,boxteal] {gauss(0,1)};
      \end{axis}
    \end{tikzpicture}

    \begin{tikzpicture}
      \node (x) at (0, 0) {$x$};
      \newGaussNode{g}{$(x) + (1,0)$};
      \node (px) at ($(g) + (1.25,0)$) {$p(x)$};
      \draw[boxdgray,edge] (x) -- (g);
      \draw[boxdgray,edge] (g) -- (px);

      \newGaussNode[label=below:$X$]{gv}{$(g) - (0,1.0)$};
      \node (xv) at ($(gv) - (1.25,0)$) {\colorbox{boxblue}{\color{white}$\mathbf{.80}$}};
      \node (pxv) at ($(gv) + (1.25,0)$) {\colorbox{boxgreen}{\color{white}$\mathbf{.29}$}};
      \draw[boxdgray,edge] (xv) -- (gv);
      \draw[boxdgray,edge] (gv) -- (pxv);
    \end{tikzpicture}
  \end{center}
\end{example}
Let $\Leaf_p$ a PC input node and denote $p$ as its inherent probability distribution. By
definition, any query $f:\mathcal{X}\to \mathcal{Y}$ which is tractable on $p$ is tractable on
$\Leaf_p$. We shall denote $\Leaf_p(\set{X})=p(\set{X})$, and often omit $p$ when its explicit form
is not needed. Evidently, a single input node lacks the expressivity for modeling complex models, otherwise we
would have just used the input distribution as a standalone model. The expressiveness of PCs comes
from recursively combining distributions into complex functions. This can be done through
computational units that either compute convex combinations or products of their children. Let us
first look at convex combinations, known in the literature as \emph{sum} nodes.

Let $\Sum$ be a PC sum node, and denote by $\Ch(\Sum)$ the children nodes of $\Sum$. For every edge
$\edge{\Sum\Child}$ coming out of $\Sum$ and going to $\Child$, we attribute a weight $w_{\Sum,
\Child}>0$, such that $\sum_{\Child\in\Ch(\Sum)} w_{\Sum,\Child} = 1$. A sum node semantically
defines a mixture model over its children, essentially acting as a latent variable over the
component distributions \citep{poon11,peharz16}. Its value is the weighted sum of its children
$p_{\Sum}(\set{X}=\set{x})=\sum_{\Child\in\Ch(\Sum)}w_{\Sum,\Child}\cdot
p_{\Child}(\set{X}=\set{x})$. To simplify notation, we shall use $\Node(\set{X}=\set{x})$ as an
alias for $p_{\Node}(\set{X}=\set{x})$, that is, the probability function given by the $\Node$'s
induced distribution.

\newcommand\mone{1}%
\newcommand\sone{0.65}%
\newcommand\mtwo{2.5}%
\newcommand\stwo{0.85}%
\newcommand\mthr{4}%
\newcommand\sthr{0.6}%
\begin{example}[sidebyside,lefthand width=0.55\textwidth]{Gaussian mixture models as probabilistic circuits}{mixgauss}
  A Gaussian Mixture Model (GMM) defines a mixture over Gaussian components. Say we wish to compute
  the probability of $X=x$ for a GMM $\mathcal{G}$ with three components
  $\gaussian_1(\mu_1=\mone,\sigma^2_1=\sone)$, $\gaussian_2(\mu_2=\mtwo, \sigma^2_1=\stwo)$ and
  $\gaussian_3(\mu_3=\mthr,\sigma^2_3=\sthr)$, and suppose we have weights set to
  $\phi=(0.4,0.25,0.35)$. Computing the probability of $G$ amounts to the weighted summation
  \begin{align*}
    \mathcal{G}(X=x)=&0.4\cdot\gaussian_1(x;\mu_1,\sigma^2_2)+0.25\cdot\gaussian_2(x;\mu_2,\sigma^2_2)+\\
                     &0.35\cdot\gaussian_3(x;\mu_3,\sigma^2_3),
  \end{align*}
  which is equivalent to a computational graph (i.e. a PC) with a sum node whose weights are set to
  $\phi$ and children are the components of the mixture. The figure on the right shows
  $\mathcal{G}$ (top) and its corresponding PC (middle). Given $x=1.5$ (in blue), input nodes are
  computed following the inference flow (bottom, gray edges) up to the root sum node (in red),
  where a weighted summation is computed to output the probability (in green).

  \tcblower
  \begin{center}
    \begin{tikzpicture}
      \pgfplotsset{
        every axis/.append style={
          axis line style={->},
          tick label style={font={\scriptsize\bfseries}},
          x tick label style={color=white,below},
          y tick label style={color=white,left},
          grid style={black,dashed},
        }
      }
      \begin{axis}[
        no markers, domain=-1:6, samples=35,
        height=3.75cm, width=\columnwidth,
        xtick={1.5}, ytick={0.2414},
        xticklabels={\colorbox{boxblue}{\textbf{1.5}}},
        yticklabels={\colorbox{boxgreen}{\textbf{0.24}}},
        axis lines*=left, xlabel=$x$, ylabel=$p(x)$,
        every axis y label/.style={font=\scriptsize,at={(axis description cs:-0.1,0.9)},anchor=south},
        every axis x label/.style={font=\scriptsize,at=(current axis.right of origin),anchor=west},
        enlargelimits=false, clip=false, axis on top,
        grid = major
      ]
        \path[name path=axis] (axis cs:0,0) -- (axis cs:5,0);
        \addplot[very thick,boxteal,name path=g1] {gauss(\mone,\sone)};
        \addplot[very thick,boxorange,name path=g2] {gauss(\mtwo,\stwo)};
        \addplot[very thick,boxpurple,name path=g3] {gauss(\mthr,\sthr)};
        \addplot[very thick,boxred] {mixgauss3(\mone,\sone,\mtwo,\stwo,\mthr,\sthr,0.4,0.25,0.35)};
        \addplot[boxteal!60] fill between [of=g1 and axis];
        \addplot[boxorange!50] fill between [of=g2 and axis];
        \addplot[boxpurple!40] fill between [of=g3 and axis];
        \node at (\mone, 0.7) {\tiny$\mu_1=\mone$};
        \node at (\mtwo, 0.5) {\tiny$\mu_2=\mtwo$};
        \node at (\mthr, 0.75) {\tiny$\mu_3=\mthr$};
      \end{axis}
    \end{tikzpicture}

    \begin{tikzpicture}
      \newSumNode[fill=boxred!70]{s}{0,0};
      \newGaussNode[fill=boxteal]{g1}{$(s) + (-1.5,-1)$};
      \newGaussNode[fill=boxorange!80]{g2}{$(s) + (0,-1)$};
      \newGaussNode[fill=boxpurple!60]{g3}{$(s) + (1.5,-1)$};
      \draw[edge] (s) edge (g1);
      \draw[edge] (s) edge (g2);
      \draw[edge] (s) edge (g3);
      \node at ($(s) + (-1.2,-0.35)$) {\scriptsize$.40$};
      \node at ($(s) + (-0.3,-0.5)$) {\scriptsize$.25$};
      \node at ($(s) + (1,-0.35)$) {\scriptsize$.35$};
      \node (l1) at ($(g1) + (0,-0.5)$) {\scriptsize$\gaussian_1(\mone,\sone)$};
      \node (l2) at ($(g2) + (0,-0.5)$) {\scriptsize$\gaussian_2(\mtwo,\stwo)$};
      \node (l3) at ($(g3) + (0,-0.5)$) {\scriptsize$\gaussian_3(\mthr,\sthr)$};
    \end{tikzpicture}

    \begin{tikzpicture}
      \newSumNode[fill=boxred!70]{s}{0,0};
      \newGaussNode[fill=boxteal]{g1}{$(s) + (-1.5,-1)$};
      \newGaussNode[fill=boxorange!80]{g2}{$(s) + (0,-1)$};
      \newGaussNode[fill=boxpurple!60]{g3}{$(s) + (1.5,-1)$};
      \draw[edge] (s) edge[bend right=5] (g1);
      \draw[edge] (s) edge[bend right=5] (g2);
      \draw[edge] (s) edge[bend right=5] (g3);
      \node at ($(s) + (-1.2,-0.35)$) {\scriptsize$.40$};
      \node at ($(s) + (-0.3,-0.5)$) {\scriptsize$.25$};
      \node at ($(s) + (1,-0.35)$) {\scriptsize$.35$};
      \node (inp) at ($(g2) + (0,-1.5)$) {\scriptsize\colorbox{boxblue}{\color{white}$\mathbf{1.5}$}};
      \node (out) at ($(s) + (0,1.0)$) {\scriptsize\colorbox{boxgreen}{\color{white}$\mathbf{0.24}$}};
      \draw[edge,boxdgray] (s) -- (out);
      \node (l1) at ($(g1) + (0,-0.5)$) {\scriptsize\colorbox{boxteal}{\color{white}$\mathbf{0.45}$}};
      \node (l2) at ($(g2) + (0,-0.5)$) {\scriptsize\colorbox{boxorange}{\color{white}$\mathbf{0.23}$}};
      \node (l3) at ($(g3) + (0,-0.5)$) {\scriptsize\colorbox{boxpurple!80}{\color{white}$\mathbf{0.00}$}};
      \draw[edge,boxdgray] (inp) edge (l1);
      \draw[edge,boxdgray] (inp) edge (l2);
      \draw[edge,boxdgray] (inp) edge (l3);
      \draw[edge,boxdgray] (g1) edge[bend left=-5] (s);
      \draw[edge,boxdgray] (g2) edge[bend left=-5] (s);
      \draw[edge,boxdgray] (g3) edge[bend left=-5] (s);
    \end{tikzpicture}
  \end{center}
\end{example}

Before we address products as computational units in PCs, we first need to discuss the concept of
scope of a PC node. Denote by $\Sc(\Node)$ the set of all variables that appear in $\Node$.  We
inductively compute the scope of circuit by a bottom-up approach: the scope of an input node
$\Leaf_p$ is the set of variables that appear in $p$'s distribution, and the scope of any other
node is the union of all of its childrens' scopes. The notion of a node's scope is essential to the
structural constraints seen in \cref{sec:const}.

So far, the only nonlinearities present in PCs come from the internal computations of input nodes.
In fact, a PC that only contains sums inputs can always be reduced to a sum node rooted PC with a
single layer, i.e. a mixture model (see \cref{thm:summix}). Adding \emph{product nodes} as another
form of nonlinearity increases expressivity sufficiently for PCs to be capable of representing any
(discrete) probability distribution \citep{darwiche03,martens14,peharz15}. More importantly,
products semantically act as factorizations of their children, indicating an independence
relationship between variables from different children. In practice, product nodes are simply
products of their childrens' distribution: if $\Prod$ is a PC product node, then its value is given
by $\Prod(\set{X}=\set{x})=\prod_{\Child\in\Ch(\Prod)}\Child(\set{X}=\set{x})$.

\newcommand\xmone{2}%
\newcommand\xsone{0.5}%
\newcommand\xmtwo{4}%
\newcommand\xstwo{0.8}%
\newcommand\ymone{3}%
\newcommand\ysone{0.7}%
\newcommand\ymtwo{5}%
\newcommand\ystwo{0.4}%
\begin{example}[sidebyside,lefthand width=0.55\textwidth]{Factors as probabilistic circuits}{factors}
  Say we have two GMMs $\mathcal{G}_1$ and $\mathcal{G}_2$. The first is a mixture model over
  variable $X$, with component weights $\phi_1=(0.3,0.7)$ and gaussians
  $\gaussian_1(\mu_1=\xmone,\sigma_1=\xsone)$ and $\gaussian_2(\mu_2=\xmtwo,\sigma_2=\xstwo)$. The
  second is composed of $\gaussian_3(\mu_3=\ymone,\sigma_2=\ysone)$ and $\gaussian_4(\mu_4=\ymtwo,
  \sigma_2=\ystwo)$, both distributions over variable $Y$ and with weights $\phi_2=(0.6,0.4)$.

  Suppose $X\indep Y$, yet we wish to compute the joint probability of both $x$ and $y$. If
  $X\indep Y$, then $p(x,y)=p(x)p(y)=\mathcal{G}_1(x)\mathcal{G}_2(y)$, which corresponds to a
  factoring of mixtures. This is represented as a product node (in green) over the two mixture
  models (in red and purple). The resulting joint of this circuit is shown below.
\begin{tikzpicture}
  \pgfplotsset{
    every axis/.append style={
      axis line style={->},
      axis lines=center,
      grid style={black,dashed},
      x tick label style={color=white,below},
      y tick label style={color=white,right},
      z tick label style={color=white,left},
    }
  }
  \begin{axis}[
    no markers, width=0.9\columnwidth,
    xtick={2}, ytick={4}, ztick={0.035},
    xticklabels={\colorbox{boxblue}{\textbf{2}}},
    yticklabels={\colorbox{boxblue}{\textbf{4}}},
    zticklabels={\colorbox{boxgreen}{\textbf{0.035}}},
    xlabel=$x$, ylabel=$y$, zlabel={$p(x,y)$},
    axis lines*=left,
    xlabel style={anchor=north west},
    ylabel style={anchor=south west},
    zlabel style={anchor=south west},
    enlargelimits=false, clip=false, axis on top,
    grid = major
  ]
    \addplot3[
      surf, samples=50,
      domain=0.5:6.5,
      y domain=1:6
    ] {((0.3*exp(-((x-\xmone)^2)/(2*\xsone^2))/\xsone+0.7*exp(-((x-\xmtwo)^2)/(2*\xstwo^2))/\xstwo)/2.5066)*(((0.6*exp(-((y-\ymone)^2)/(2*\ysone^2))/\ysone+0.4*exp(-((y-\ymtwo)^2)/(2*\ystwo^2))/\ystwo)/2.5066))};
  \end{axis}
\end{tikzpicture}

  \tcblower
  \begin{center}
    \begin{tikzpicture}
      \pgfplotsset{
        every axis/.append style={
          axis line style={->},
          tick label style={font={\scriptsize\bfseries}},
          x tick label style={color=white,below},
          y tick label style={color=white,left},
          grid style={black,dashed},
        }
      }
      \begin{axis}[
        no markers, domain=0:7, samples=35,
        height=3.0cm, width=\columnwidth,
        xtick={2}, ytick={0.25},
        xticklabels={\colorbox{boxblue}{\textbf{2}}},
        yticklabels={\colorbox{boxred}{\textbf{0.25}}},
        axis lines*=left, xlabel=$x$, ylabel=$p(x)$,
        every axis y label/.style={font=\scriptsize,at={(axis description cs:-0.1,0.9)},anchor=south},
        every axis x label/.style={font=\scriptsize,at=(current axis.right of origin),anchor=west},
        enlargelimits=false, clip=false, axis on top,
        grid = major
      ]
        \path[name path=axis] (axis cs:0,0) -- (axis cs:7,0);
        \addplot[very thick,boxteal,name path=g1] {gauss(\xmone,\xsone)};
        \addplot[very thick,boxorange,name path=g2] {gauss(\xmtwo,\xstwo)};
        \addplot[very thick,boxred] {mixgauss2(\xmone,\xsone,\xmtwo,\xstwo,0.3,0.7)};
        \addplot[boxteal!60] fill between [of=g1 and axis];
        \addplot[boxorange!50] fill between [of=g2 and axis];
        \node at (axis cs:\xmone,{egauss(\xmone,\xsone,\xmone)+0.1}) {\tiny$\mu_1=\xmone$};
        \node at (axis cs:\xmtwo,{egauss(\xmtwo,\xstwo,\xmtwo)+0.1}) {\tiny$\mu_2=\xmtwo$};
      \end{axis}
    \end{tikzpicture}
    \begin{tikzpicture}
      \pgfplotsset{
        every axis/.append style={
          axis line style={->},
          tick label style={font={\scriptsize\bfseries}},
          x tick label style={color=white,below},
          y tick label style={color=white,left},
          grid style={black,dashed},
        }
      }
      \begin{axis}[
        no markers, domain=1:7, samples=50,
        height=3.0cm, width=\columnwidth,
        xtick={4}, ytick={0.14},
        xticklabels={\colorbox{boxblue}{\textbf{4}}},
        yticklabels={\colorbox{boxpurple}{\textbf{0.14}}},
        axis lines*=left, xlabel=$y$, ylabel=$p(y)$,
        every axis y label/.style={font=\scriptsize,at={(axis description cs:-0.1,0.9)},anchor=south},
        every axis x label/.style={font=\scriptsize,at=(current axis.right of origin),anchor=west},
        enlargelimits=false, clip=false, axis on top,
        grid = major
      ]
        \path[name path=axis] (axis cs:0,0) -- (axis cs:7,0);
        \addplot[very thick,boxpink,name path=g1] {gauss(\ymone,\ysone)};
        \addplot[very thick,boxgoldenrod,name path=g2] {gauss(\ymtwo,\ystwo)};
        \addplot[very thick,boxpurple] {mixgauss2(\ymone,\ysone,\ymtwo,\ystwo,0.6,0.4)};
        \addplot[boxpink!40] fill between [of=g1 and axis];
        \addplot[boxgoldenrod!50] fill between [of=g2 and axis];
        \node at (axis cs:\ymone,{egauss(\ymone,\ysone,\ymone)+0.1}) {\tiny$\mu_3=\ymone$};
        \node at (axis cs:\ymtwo,{egauss(\ymtwo,\ystwo,\ymtwo)+0.1}) {\tiny$\mu_4=\ymtwo$};
      \end{axis}
    \end{tikzpicture}
    \begin{tikzpicture}
      \newProdNode[fill=boxgreen]{r}{0,0};
      \newSumNode[fill=boxred!70]{p}{$(r) + (-1.25,-0.75)$};
      \newSumNode[fill=boxpurple!60]{q}{$(r) + (1.25,-0.75)$};
      \newGaussNode[fill=boxteal]{x1}{$(p) + (-0.6,-1)$};
      \newGaussNode[fill=boxorange!80]{x2}{$(p) + (0.65,-1)$};
      \newGaussNode[fill=boxpink!50]{y1}{$(q) + (-0.6,-1)$};
      \newGaussNode[fill=boxgoldenrod!70]{y2}{$(q) + (0.6,-1)$};
      \draw[edge] (r) edge (p);
      \draw[edge] (r) edge (q);
      \draw[edge] (p) edge (x1);
      \draw[edge] (p) edge (x2);
      \draw[edge] (q) edge (y1);
      \draw[edge] (q) edge (y2);
      \node at ($(p) + (-0.5,-0.4)$) {\scriptsize$.3$};
      \node at ($(p) + (0.5,-0.4)$) {\scriptsize$.7$};
      \node at ($(q) + (-0.5,-0.4)$) {\scriptsize$.6$};
      \node at ($(q) + (0.5,-0.4)$) {\scriptsize$.4$};
      \node (l1) at ($(x1) + (0,-0.5)$) {\scriptsize$\gaussian_1(\xmone,\xsone)$};
      \node (l2) at ($(x2) + (0,-0.5)$) {\scriptsize$\gaussian_2(\xmtwo,\xstwo)$};
      \node (l3) at ($(y1) + (0,-0.5)$) {\scriptsize$\gaussian_1(\ymone,\ysone)$};
      \node (l4) at ($(y2) + (0,-0.5)$) {\scriptsize$\gaussian_2(\ymtwo,\ystwo)$};
    \end{tikzpicture}
    \begin{tikzpicture}
      \newProdNode[fill=boxgreen]{r}{0,0};
      \newSumNode[fill=boxred!70]{p}{$(r) + (-1.25,-0.75)$};
      \newSumNode[fill=boxpurple!60]{q}{$(r) + (1.25,-0.75)$};
      \newGaussNode[fill=boxteal]{x1}{$(p) + (-0.6,-1)$};
      \newGaussNode[fill=boxorange!80]{x2}{$(p) + (0.65,-1)$};
      \newGaussNode[fill=boxpink!50]{y1}{$(q) + (-0.6,-1)$};
      \newGaussNode[fill=boxgoldenrod!70]{y2}{$(q) + (0.6,-1)$};
      \draw[edge] (r) edge[bend right=5] (p);
      \draw[edge] (r) edge[bend right=5] (q);
      \draw[edge] (p) edge[bend right=5] (x1);
      \draw[edge] (p) edge[bend right=5] (x2);
      \draw[edge] (q) edge[bend right=5] (y1);
      \draw[edge] (q) edge[bend right=5] (y2);
      \node at ($(p) + (-0.5,-0.4)$) {\scriptsize$.3$};
      \node at ($(p) + (0.5,-0.4)$) {\scriptsize$.7$};
      \node at ($(q) + (-0.5,-0.4)$) {\scriptsize$.6$};
      \node at ($(q) + (0.5,-0.4)$) {\scriptsize$.4$};
      \node (l1) at ($(x1) + (0,-0.5)$) {\scriptsize\colorbox{boxteal}{\color{white}$\mathbf{0.24}$}};
      \node (l2) at ($(x2) + (0,-0.5)$) {\scriptsize\colorbox{boxorange}{\color{white}$\mathbf{0.01}$}};
      \node (l3) at ($(y1) + (0,-0.5)$) {\scriptsize\colorbox{boxpink}{\color{white}$\mathbf{0.12}$}};
      \node (l4) at ($(y2) + (0,-0.5)$) {\scriptsize\colorbox{boxgoldenrod}{\color{white}$\mathbf{0.01}$}};
      \draw[edge,boxdgray] (p) edge[bend left=-5] (r);
      \draw[edge,boxdgray] (q) edge[bend left=-5] (r);
      \draw[edge,boxdgray] (x1) edge[bend left=-5] (p);
      \draw[edge,boxdgray] (x2) edge[bend left=-5] (p);
      \draw[edge,boxdgray] (y1) edge[bend left=-5] (q);
      \draw[edge,boxdgray] (y2) edge[bend left=-5] (q);
      \node at ($(p) + (-0.3,0.6)$) {\scriptsize\colorbox{boxred}{\color{white}$\mathbf{0.25}$}};
      \node at ($(q) + (0.3,0.6)$) {\scriptsize\colorbox{boxpurple}{\color{white}$\mathbf{0.14}$}};
      \node (x) at ($(p) + (0,-2.5)$) {\scriptsize\colorbox{boxblue}{\color{white}$\mathbf{x=2}$}};
      \node (y) at ($(q) + (0,-2.5)$) {\scriptsize\colorbox{boxblue}{\color{white}$\mathbf{y=4}$}};
      \draw[edge,boxdgray] (x) edge (l1);
      \draw[edge,boxdgray] (x) edge (l2);
      \draw[edge,boxdgray] (y) edge (l3);
      \draw[edge,boxdgray] (y) edge (l4);
      \node (out) at ($(r) + (0,0.9)$) {\scriptsize\colorbox{boxgreen}{\color{white}$\mathbf{0.035}$}};
      \draw[edge,boxdgray] (r) edge (out);
    \end{tikzpicture}
  \end{center}
\end{example}

\begin{remark}[breakable]{On operators and tractability}{optract}
  Throughout this work we consider only products and convex combinations (apart from the implicit
  operations contained within input nodes) as potential computational units. The question of whether
  any other operator could be used to gain expressivity without loss of tractability is without a
  doubt an interesting research question, and one that is actively being pursued. However, this is
  certainly out of the scope of this dissertation, and so we restrict discussion on this topic and
  only give a brief comment on operator tractability here, pointing to existing literature in this
  area of research.

  \citet{friesen16} formalize the notion of replacing sums and products in PCs with any pair of
  operators in a commutative semiring, giving results on the conditions for marginalization to be
  tractable. They provide examples of common semirings and to which known formalisms they
  correspond to. One such example are PCs under the Boolean semiring $(\{0,1\},\vee,\wedge,0,1)$
  for logical inference, which are equivalent to Negation Normal Form (NNF, \cite{barwise82}) and
  constitute an instance of Logic Circuits (LCs), of which Sentential Decision Diagrams (SDDs,
  \cite{darwiche11}) and Binary Decision Diagrams (BDDs, \cite{akers78}) are a part of. Another
  less common semiring in PCs is the real min-sum semiring $(\mathbb{R}_{\infty}, \min,+,\infty,0)$
  for nonconvex optimization \citep{friesen15}.

  Recently, \citet{vergari21} extensively covered tractability conditions and complexity bounds for
  convex combinations, products, $\exp$ (and more generally powers in both naturals and reals),
  quotients and logarithms, even giving results for complex information-theoretic queries, such as
  entropies and divergences. Notably, they analyze whether structural constraints (and thus, in a
  sense, tractability) under these conditions are preserved.

  Up to now, we have only considered summations as nonnegative weighted sums. Indeed, in most
  literature the sum node is defined as a convex combination. However, negative weights have
  appeared in Logistic Circuits \citep{liang19} for discriminative modeling; and in Probabilistic
  Generating Circuits \citep{zhang21}, a class of tractable probabilistic models that subsume PCs.
  \citet{maua17a} extend (nonnegative) weights in sum nodes with probability intervals, effectively
  inducing a credal set \citep{cozman00} for measuring imprecision.

  Other works include PCs with quotients \citep{sharir18a}, transformations \citep{pevny20a}, max
  \citep{melibari16}, and einsum \citep{peharz20b} operations.
\end{remark}

\section{Deciding What to Constraint}
\label{sec:const}


referenced structural constraints in passing

