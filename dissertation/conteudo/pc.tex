\chapter{Probabilistic Circuits}
\label{ch:pc}

As we briefly mentioned in the last chapter, Probabilistic Circuits (PCs) are conceptualized as
computational graphs under special conditions. In this chapter, \cref{sec:pc} to be more precise,
we formally define PCs and give an intuition on their syntax, viewing other probabilistic models
through the lenses of the PC framework. In \cref{sec:const}, we describe the special structural
constraints that give PCs their inference power over other generative models and state which
queries (as far as we know) are enabled from each constraint.

\section{Distributions as Computational Graphs}
\label{sec:pc}

In its simplest form, a probabilistic circuit is a single computational unit corresponding to some
distribution. Following Graph Theory nomenclature, computational units with no  These can range from univariate or multivariate exponential distributions to complex
non-parametric models. To simplify notation, when 

$\gaussian$.

\section{Deciding What to Constraint}
\label{sec:const}


referenced structural constraints in passing

