\chapter{A Data Perspective to Scalable Learning}
\label{ch:data}

We now turn our attention to scalably learning a PC purely from data. In this chapter, we look at
PCs solely from the perspective of data fitness; we exploit the connection between PCs and
generative random forests \citep{correia20,ho95} and revisit a well-known technique based on random
projections for constructing random trees \citep{dasgupta08a,dasgupta08b}, presenting a simple and
fast yet effective way of learning PCs. This approach learns smooth and structure decomposable
circuits by randomly partitioning the data space with random projections in a \divclass{} class
fashion. We show that our method produces competitive PCs at a fraction of the time. The
contributions in this chapter come, in part, from \citet{geh21b}.

\section{Probabilistic Circuits and Decision Trees}
\label{sec:data}

Before we go through with our proposal in detail, we must first lay the groundwork and motivate the
decisions behind our structure learning algorithm. We begin by re-emphasizing the connection
between probabilistic circuits and density estimation trees briefly discussed in \Cref{eg:det}. We
follow by revisiting \emph{random projections} \citep{dasgupta08a,dasgupta08b}, a well-known
technique for hierarchically partitioning data through oblique hyperplanes. Next, we present in
detail a very fast structure learning algorithm for quickly generating smooth and structure
decomposable circuits. Despite their simplicity, we empirically show their competitive performance
compared to state-of-the-art.

Recently, \citet{correia20} showed that (ensembles of) decision trees (DTs) learned for prediction
tasks can be easily extended into full probabilistic models represented as probabilistic circuits.
Besides equipping decision forests with more principles approaches to handling missing data and
diagnosing outliers, this bridge between decision trees and PCs suggests an interesting alternative
to learning the latter using the efficient inductive algorithms available for the former
\citep{correia20,ram11,khosravi20}. Despite this, most works addressing such a connection have
focused on the discriminative side of DTs, with much of the effort put onto classification rather
than generative tasks such as density estimation. Here, we explore the generative side of DTs,
often referred as density estimation trees (DETs, \cite{ram11,hang19,smyth95}), within the framework
of PCs, taking inspiration from known algorithms for building DTs and DETs, and transplanting them
to PCs. 

As shortly discussed in \Cref{eg:det}, a DET can be represented as a smooth and deterministic PC
with only sums and input nodes, the latter's supports restricted to the data cells induced by the
partitioning data hyperplanes. The resulting density of this DET PC is given by
\begin{equation}
  p_\mathcal{C}(\set{x})=\sum_{\Leaf\in\operatorname{Inputs}(\mathcal{C})}w_{\Leaf}\cdot\Leaf(\set{x})\cdot\liv\set{x}\in\Leaf\riv,
\end{equation}
where $\liv\set{x}\in\Leaf\riv$ is an indicator function that returns 1 if $\set{x}$ is within
$\Leaf$'s cell and 0 otherwise. The above formula comes from collapsing a circuit with only sum
layers, where each sum corresponds to a latent variable representing a partitioning of the data,
into a single-layer shallow PC. These latent variables usually consist of partitioning data through
hyperplanes, dividing data into two parts, each represented as the subcircuit rooted at each child
of the sum node.

A $k$-d tree is a subclass of decision trees which hierarchically partitions data into more or less
equally sized parts, usually by splitting the data according to the value of a single variable at a
time \citep{bentley75,hang19,ho95}, essentially producing axis-aligned hyperplanes.
\citet{dasgupta08b} noted that such an approach cannot ensure that the resulting partitioning of
the input space approximates the intrinsic dimensionality of data (roughly understood as a manifold
of low dimension). In contrast, they provide a simple strategy for space partitioning that consists
in recursively partitioning the space according to a random separating hyperplane. This
approximates a random projection of the data and has the following theoretical guarantee
\citep{dasgupta08b}:

\begin{figure}[t]
  \def\points{(2.3,4.38),(1.8,4.0),(2.8,4.0),(1.6,3.3),(2.5,3),(3.1,2.8),(3.8,3.1),(4.3,3.3),(4,2.6),(4.5,2.4),(3.3,2.3),(2.1,2.0),(3.1,1.8),(2.4,1.6),(2.9,1.5),(2.5,1.3),(3.1,1.1),(3.1,0.6),(2.7,0.8),(1.5,0.8),(1.7,1.3),(1.2,1.5),(3.6,0.5),(4,0.3),(4.1,0.6),(3.8,0.8),(4.4,0.8),(4,1.1),(4.3,1.1),(4.6,1),(3.6,1.2),(3.9,1.4),(4.2,1.4),(4.6,1.4),(4.4,1.7),(3.8,1.7)}
  \begin{subfigure}[t]{0.495\textwidth}
    \centering
    \begin{tikzpicture}
      \foreach \p in \points {
        \node[circle,inner sep=0pt,minimum size=2pt,fill=boxgray] at \p {};
      }
      \draw[very thick,boxblue,dashed] (3.5,0) -- node[left,pos=0.9] {\color{boxblue}$\set{A}$} (3.5,5);
      \draw[very thick,boxgreen,dashed] (0,2.5) -- node[right,pos=1] {\color{boxgreen}$\set{B}$} (3.5,2.5);
      \draw[very thick,boxpurple!80,dashed] (2,0) -- node[above,pos=1] {\color{boxpurple}$\set{C}$} (2,2.5);
      \draw[very thick,boxmunsel,dashed] (0,3.75) -- node[right,pos=1] {\color{boxmunsel}$\set{D}$} (3.5,3.75);
      \draw[very thick,boxolive!80!black,dashed] (3.5,2) -- node[below,pos=0.5] {\color{boxolive!80!black}$\set{E}$} (5,2);
      \draw[very thick] (0,0) rectangle (5, 5);
    \end{tikzpicture}
    \caption{Axis-aligned projections}
  \end{subfigure}
  \begin{subfigure}[t]{0.495\textwidth}
    \centering
    \begin{tikzpicture}
      \foreach \p in \points {
        \node[circle,inner sep=0pt,minimum size=2pt,fill=boxgray] at \p {};
      }
      \draw[very thick,boxblue,dashed] (2.2,0) coordinate (a1) -- node[left,pos=0.9] {\color{boxblue}$\set{A}$} (5,3.1) coordinate (a2);
      \draw[very thick,boxgreen,dashed] ($(a1)!0.5!(a2)$) coordinate (b1) -- node[above right,pos=0.9] {\color{boxgreen}$\set{B}$} (0,4.2) coordinate (b2);
      \draw[very thick,boxpurple!80,dashed] ($(b1)!0.35!(b2)$) coordinate (c1) -- node[above left,pos=0.7] {\color{boxpurple}$\set{C}$} (5,5) coordinate (c2);
      \draw[very thick,boxmunsel,dashed] ($(b1)!0.22!(b2)$) coordinate (d1) -- node[above left,pos=0.9] {\color{boxmunsel}$\set{D}$} (1.8,0) coordinate (d2);
      \draw[very thick,boxolive!80!black,dashed] ($(a1)!0.45!(a2)$) -- node[right,pos=0.6] {\color{boxolive!80!black}$\set{E}$} (5,0);
      \draw[very thick] (0,0) rectangle (5, 5);
    \end{tikzpicture}
    \caption{Random projections}
  \end{subfigure}
  \caption{Two partitionings induced by $k$-d trees: \textbf{(a)} shows axis-aligned splits and
    \textbf{(b)} random projection splits. Gray dots are datapoints, dashed lines are (hyper)planes.}
    \label{fig:projections}
\end{figure}

\begin{quote}
  If the data has intrinsic dimension $d$, then with constant probability the part of the data at
  level $d$ or higher of the tree has average diameter less than half of the data.
\end{quote}

Accordingly, the depth of the tree needs only to grow proportionally to the intrinsic dimension and
not to the number of variables. In addition to that and to other theoretical insurances
\citep{dhesi10}, the recursive partitioning scheme proposed is extremely fast, taking linearithmic
time in the dataset size (number of instances and variables). \citet{dasgupta08a} further
empirically show that employing random projections boosts performance significantly compared to
regular axis-aligned projections.

\section{Random Projections}
\label{sec:rp}

Let $\set{D}$ be a dataset with $\set{X}$ variables. A function $f:\mathcal{X}\to\{0,1\}$ describes a
hyperplane over variables in $\set{X}$ and is here called a \emph{rule}. A rule partitions data by
assigning observations to either one, $\set{S}_1=\{\set{x}\,|\,\forall\set{x}\in\set{D}\wedge
f(\set{x})=0\}$, or the other, $\set{S}_2=\{\set{x}\,|\,\forall\set{x}\in\set{D}\wedge
f(\set{x}=1)\}$, of two data partitions. In $k$-d trees, this partitioning proceeds recursively
until $|\set{D}|$ is sufficiently small. When employing axis-aligned partitions, $f$ typically
selects the variable with the largest variance (or some other measure of spread) in $\set{D}$ and
separates instance according to the median value of that variable. The process is similar to the
induction of decision trees, except that in this case the rules discriminate against a target
variable \citep{breiman01}.

\begin{algorithm}[t]
  \caption{\textproc{SplitSID}}\label{alg:splitsid}
  \begin{algorithmic}[1]
    \Require Dataset $\set{D}\subset\mathbb{R}^m$
    \Ensure A partition $\set{S}_1,\set{S}_2$ of $\set{D}$
    \State Let $n$ be the number of examples in $\set{D}$
    \State Sample a random unit direction $\set{w}$
    \State Sort $\set{b}=\set{a}\cdot\set{x}$ for $\set{x}\in\set{D}$ s.t.\ $b_1\leq b_2\leq\dotsb\leq b_n$
    \For{each $i\in\left[n-1\right]$}
      \State $\mu_1 = \frac{1}{i}\sum_{j=1}^i b_i$, \; $\mu_2 = \frac{1}{n-i}\sum_{j=i+1}^n b_i$
      \State $c_i = \sum_{j=1}^i (b_j - \mu_1)^2 + \sum_{j+1}^n (b_j - \mu_2)^2$
    \EndFor
    \State Find $i$ that minimizes $c_i$ and set $\theta = (b_i + b_{i+1})/2$
    \State $\set{S}_1 \gets \{\set{x}\,|\,\forall\set{x}\in\set{D}\wedge\set{a}\cdot\set{x}\leq\theta\}$
    \State \textbf{return} $(\set{S}_1,\set{D} \setminus \set{S}_1)$
  \end{algorithmic}
\end{algorithm}

\begin{algorithm}[t]
  \caption{\textproc{SplitMax}}\label{alg:splitmax}
  \begin{algorithmic}[1]
    \Require Dataset $\set{D} \subset \mathbb{R}^m$ and constant $r$
    \Ensure A partition $(\set{S}_1,\set{S}_2)$ of $\set{D}$
    \State Sample a random unit direction $\set{a}$
    \State Pick any $\set{x}\in\set{D}$ and let $\set{y}$ be $\set{x}$'s farthest point in $\set{D}$
    \State Sample $\delta$ uniformly in $[-c,c]$, where ${c=r\cdot\operatorname{dist}(\set{x},\set{y})/\sqrt{m}}$
    \State $\set{S}_1\gets\{\set{x}\,|\,\forall\set{x}\in\set{D}\wedge\set{a}\cdot\set{x}\leq\operatorname{median}(\{\set{z}\cdot\set{a}\,|\,\set{z}\in\set{D}\})+\delta\}$
    \State \textbf{return} $(\set{S}_1,\set{D}\setminus\set{S}_1)$
  \end{algorithmic}
\end{algorithm}

The statistical properties of estimates obtained from the instances at the leaves of a $k$-d tree
depend on the rate at which the diameter of the partitions are reduced once we move down the tree.
For a space of dimension $m$, a $k$-d tree induced by the process described might require $m$
levels to halve the diameter of the original data \citep{dasgupta08b}. This is true even for
datasets of low intrinsic dimension. The latter is variously defined, and different definitions
lead to different theoretical properties. A common surrogate metric is the \emph{doubling
dimension} of the dataset $\set{D}\subset\mathbb{R}^m$, given by the smallest integer $d$ such that
the intersection of $\set{D}$ and any ball of radius $r$ centered at $\set{x}\in\set{D}$ can be
covered by at most $2^d$ balls of radius $\frac{r}{2}$ \citep{dhesi10}.

Random Projection Trees (RPTrees) are a special type of $k$-d trees that split along a random
direction of the space. Two such splitting rules are given by \Cref{alg:splitsid,alg:splitmax},
where in the latter $\operatorname{dist}(\set{x},\set{y})$ refers to the Euclidean distance. The
intuition behind either rule is to generate a random hyperplane (unit direction) and then find a
threshold projection value that roughly divides dataset $\set{D}$ into two approximately equal
sized subsets. More concretely, we wish to find a mapping $f:\set{x}\mapsto\left\liv\sum_{i=1}^d
x_i\cdot a_i>\theta\right\riv$ that discriminates an assignment $\set{x}$ to one or the other side
of a hyperplane described as the random unit vector $\set{a}=\left(a_1,\ldots,a_d\right)$.
\Cref{alg:splitsid} attempts at finding the projection threshold $\theta$ by minimizing the average
squared interpoint distance (SID), while \Cref{alg:splitmax} uses a random noise proportional to
the average diameter of $\set{D}$. As discussed by \citet{dasgupta08b} and by \citet{dasgupta08a},
either optimizing or randomizing the threshold leads to better separation of data than simply
selecting the median point. \Cref{fig:projections} shows the difference between axis-aligned and
random projections, while \Cref{fig:rps} shows an example of space partitioning induced by 2-level
RPTrees using each of the rules with the same direction vectors $w$. Note that the rules produce
quite different splits despite using the same random directions.

\begin{figure}[t]
  \centering
  \begin{subfigure}[t]{0.495\textwidth}
    \centering
    \begin{tikzpicture}
      \begin{axis}[
        title={\textproc{SplitMax}},
        tick style={draw=none},
        ytick=\empty,
        xtick=\empty,
        enlargelimits=0,
        ymin=-1.8,
        ymax=1.8,
        xmin=-1.8,
        xmax=1.8,
      ]
        \addplot[only marks, olive, opacity=0.5] table {data/max_1_full_parts.data};
        \addplot[only marks, teal, opacity=0.5] table {data/max_2_full_parts.data};
        \addplot[only marks, magenta, opacity=0.5] table {data/max_3_full_parts.data};
        \addplot[only marks, violet, opacity=0.5] table {data/max_4_full_parts.data};
        \addplot[black, thick] coordinates {
          (-0.7435410726460153, 0.2259926265097693)
          (2, -1.906932050205991)
        };
        \addplot[black, thick] coordinates {
          (-0.5739178598189635, 0.33907476839447037)
          (-2, 2.4781979786660253)
        };
        \addplot[black,thick] coordinates {
          (-2, -0.6116466583928871)
          (2, 2.0550200082737793)
        };
      \end{axis}
    \end{tikzpicture}
  \end{subfigure}
  \begin{subfigure}[t]{0.495\textwidth}
    \centering
    \begin{tikzpicture}
      \begin{axis}[
        title={\textsc{SplitSID}},
        tick style={draw=none},
        ytick=\empty,
        xtick=\empty,
        enlargelimits=0,
        ymin=-1.8,
        ymax=1.8,
        xmin=-1.8,
        xmax=1.8
      ]
        \addplot[only marks, olive, opacity=0.5] table {data/sid_1_full_parts.data};
        \addplot[only marks, teal, opacity=0.5] table {data/sid_2_full_parts.data};
        \addplot[only marks, magenta, opacity=0.5] table {data/sid_3_full_parts.data};
        \addplot[only marks, violet, opacity=0.5] table {data/sid_4_full_parts.data};
        \addplot[black, thick] coordinates {
        (-2, -1.3212366499672479)
        (2, 1.3454300166994186)
        };
        \addplot[black, thick] coordinates {
        (-2, 2.346171790525994)
        (-0.3073499505415808, -0.19280328366163507)
        };
        \addplot[black,thick] coordinates {
        (-0.023469760285374355, -0.003549823490830769)
        (2, -1.576665905962419)
        };
      \end{axis}
    \end{tikzpicture}
  \end{subfigure}
  \caption{Example of space partitioning by RPTrees grown using different split rules but the same random directions.}
  \label{fig:rps}
\end{figure}

Unlike standard $k$-d trees, RPTrees ensure that, for a data with doubling dimension $d$, at most
$d$ levels are necessary to half the diameter of the data, irrespective of its dimension. This
leads to improved statistical properties that are connected to that notion of low intrinsic
dimensionality \citep{dasgupta08b,dhesi10}. Inspired by these findings, we propose a fast
randomized structure learning algorithm for learning probabilistic circuits by recursively stacking
random projections in a divide-and-conquer manner similar to what is done in \textproc{LearnSPN}.
Albeit our contributions are minor, we found that our approach is extremely fast and reaches
competitive performance in binary benchmark datasets.

\section{\textproc{LearnRP}}

A random projection (RP) naturally induces a clustering of data: given a rule $f$, two clusters are
formed from the partitions induced by $f$'s hyperplane. We use this simple yet extremely fast
clustering method to replace clustering techniques in \textproc{LearnSPN}. We justify this move
from a theoretical and practical aspect. From a theoretical perspective, every sum node created
this way defines a latent variable corresponding to a hyperplane, giving some interpretability
(akin to decision trees) to the model. From a more practical point of view, we point to the work
of \citet{vergari15}, showing that binary partitions (both row-wise and column-wise) favor a deeper
architecture and produce smaller models. We further strengthen this last point by restricting the
learned structure to a vtree, effectively constructing smooth and structure decomposable PCs.

\begin{algorithm}[t]
  \caption{\textproc{LearnRP}}\label{alg:learnrp}
  \begin{algorithmic}[1]
    \Require Dataset $\set{D}$, variables $\set{X}$, vtree $\vtree$ and $k$ projection tryouts
    \Ensure A smooth and structure decomposable probabilistic circuit
    \IIf{$|\set{X}|=1$}{\textbf{return} an input node learned from $\set{D}$}\label{alg:learnrp:line:input}
    \NIElse
      \State Sample $k$ projections and call $f$ the one which minimizes the avg. diameter of $\set{D}$
      \State $\set{S}_1\gets\{\set{x}\,|\,\forall\set{x}\in\set{D}\wedge f(\set{x})=1\}$,
        $\set{S}_2=\{\set{x}\,|\,\forall\set{x}\in\set{D}\wedge f(\set{x})=0\}$
        \State Let $v$ the root of $\vtree$ \label{alg:learnrp:line:vtree}
      \State $\Child^{(1)}_1\gets\Call{LearnRP}{\set{S}_1,\Sc(v^\gets),v^\gets,k}$
      \State $\Child^{(1)}_2\gets\Call{LearnRP}{\set{S}_1,\Sc(v^\to),v^\to,k}$
      \State $\Child^{(2)}_1\gets\Call{LearnRP}{\set{S}_2,\Sc(v^\gets),v^\gets,k}$
      \State $\Child^{(2)}_2\gets\Call{LearnRP}{\set{S}_2,\Sc(v^\to),v^\to,k}$
      \State Construct products $\Prod_1\gets\Child^{(1)}_1\cdot\Child^{(1)}_2$ and
        $\Prod_2\gets\Child^{(2)}_1\cdot\Child^{(2)}_2$\label{alg:learnrp:line:prod}
      \State \textbf{return} sum $\frac{\set{S}_1}{\set{D}}\cdot\Prod_1+\frac{\set{S}_2}{\set{D}}\cdot\Prod_2$
    \EndNIElse
  \end{algorithmic}
\end{algorithm}

Of note is the fact that DETs are deterministic by nature, as the leaves of the binary tree are
constrained to the corresponding cells, meaning that only one path from leaf to root is active at a
time. Here, determinism could be enforced if, for every input node (line
\ref{alg:learnrp:line:input} in \Cref{alg:learnrp}) its support is truncated to the cell induced by
all random projections above it, similar to what is done when representing DETs as PCs. For the
discrete case, we might attribute zero mass to assignments outside its cell, normalizing the
distribution with the remaining mass. However, apart from the fact that doing so is not so trivial
in the general case (i.e.\ in the continuous) as the projections are oblique and the resulting
truncated distributions must have tractable marginalization, this process also violates
decomposability: by truncating inputs, we are essentially turning the previously univariate inputs
from line \ref{alg:learnrp:line:input} into multivariate distributions covering the entire scope,
making products learned in line \ref{alg:learnrp:line:prod} nondecomposable. This comes from the
fact that each hyperplane is a function $f(\set{x})=\left\liv\sum_{i=1}^d x_i\cdot
a_i>\theta\right\riv$ with $\Sc(f)=\Sc(v)$, where $v$ is the vtree node of line
\ref{alg:learnrp:line:vtree}. Further, because each variable $X\in\Sc(v)$ contributes (linearly) to
$f$, marginalization in this multivariate distribution is not as straightforward. We therefore
choose not to constraint inputs.

\section{Experiments}

For binary data, we evaluate \textproc{LearnRP} on the 20 well-known binary datasets for density
estimation \citep{lowd10,haaren12}\footnote{Taken from
\url{https://github.com/UCLA-StarAI/Density-Estimation-Datasets}.} and compare against reported
results from \textproc{LearnSPN} \citep{gens13}, \textproc{Strudel}, \textproc{LearnPSDD} (both
from the benchmarks reported in \cite{dang20}), \textproc{Prometheus} \citep{jaini18a} and
\textproc{XPC} \citep{dimauro21}. To measure performance in the continuous, we compare
\textproc{LearnRP} against the performance of \textproc{Prometheus}, deep Boltzmann machines (SRBMs,
\cite{salakhutdinov09}), an offline version of \citeauthor{hsu17}'s online structure learning with
Gaussian leaves (oSLRAU, \cite{hsu17}), Gaussian mixture models with Bayesian moment matching
(GBMMs, \cite{jaini16}) and infinite sum-product trees (iSPTs, \cite{trapp16}) all of which are
reported in \citet{jaini18a}. We used the same 10 continuous datasets as
\citet{jaini18a}\footnote{Which we compiled to \url{https://github.com/RenatoGeh/CDEBD}.}, which
although are reported as coming from the UCI Machine Learning Repository \citep{dua17} and Bilkent
University's Function Approximation Repository \citep{guvenir00}, are numerically distinct from the
ones found in these repositories. \Cref{tab:details} shows detailed information of every dataset
evaluated.

\begin{table}[t]
  \resizebox{\textwidth}{!}{
  \begin{tabular}{c|cccc||c|cccc}
    \hline
    \textbf{Dataset} & \textbf{Vars} & \textbf{Train} & \textbf{Test} & \textbf{Domain} &
    \textbf{Dataset} & \textbf{Vars} & \textbf{Train} & \textbf{Test} & \textbf{Domain}\\
    \hline
    \textsc{accidents } & 111  & 12758   & 2551   & $\{0,1\}$ & \textsc{nltcs     } & 16  & 16181    & 3236    & $\{0,1\}$ \\
    \textsc{ad        } & 1556 & 2461    & 491    & $\{0,1\}$ & \textsc{plants    } & 69  & 17412    & 3482    & $\{0,1\}$ \\
    \textsc{audio     } & 100  & 15000   & 3000   & $\{0,1\}$ & \textsc{pumsb-star} & 163 & 12262    & 2452    & $\{0,1\}$ \\
    \textsc{bbc       } & 1058 & 1670    & 330    & $\{0,1\}$ & \textsc{eachmovie } & 500 & 4524     & 591     & $\{0,1\}$ \\
    \textsc{netflix   } & 100  & 15000   & 3000   & $\{0,1\}$ & \textsc{retail    } & 135 & 22041    & 4408    & $\{0,1\}$ \\
    \textsc{book      } & 500  & 8700    & 1739   & $\{0,1\}$ & \textsc{abalone   } & 8   & 3760     & 417     & $\mathbb{R}$ \\
    \textsc{20-newsgrp} & 910  & 11293   & 3764   & $\{0,1\}$ & \textsc{ca        } & 22  & 7373     & 819     & $\mathbb{R}$ \\
    \textsc{reuters-52} & 889  & 6532    & 1540   & $\{0,1\}$ & \textsc{quake     } & 4   & 1961     & 217     & $\mathbb{R}$ \\
    \textsc{webkb     } & 839  & 2803    & 838    & $\{0,1\}$ & \textsc{sensorless} & 48  & 52659    & 5850    & $\mathbb{R}$ \\
    \textsc{dna       } & 180  & 1600    & 1186   & $\{0,1\}$ & \textsc{banknote  } & 4   & 1235     & 137     & $\mathbb{R}$ \\
    \textsc{jester    } & 100  & 9000    & 4116   & $\{0,1\}$ & \textsc{flowsize  } & 3   & 1358674  & 150963  & $\mathbb{R}$ \\
    \textsc{kdd       } & 65   & 180092  & 34955  & $\{0,1\}$ & \textsc{kinematics} & 8   & 7373     & 819     & $\mathbb{R}$ \\
    \textsc{kosarek   } & 190  & 33375   & 6675   & $\{0,1\}$ & \textsc{iris      } & 4   & 90       & 10      & $\mathbb{R}$ \\
    \textsc{msnbc     } & 17   & 291326  & 58265  & $\{0,1\}$ & \textsc{oldfaith  } & 2   & 245      & 27      & $\mathbb{R}$ \\
    \textsc{msweb     } & 294  & 29441   & 5000   & $\{0,1\}$ & \textsc{chemdiabet} & 3   & 131      & 14      & $\mathbb{R}$ \\
    \hline
  \end{tabular}
  }
  \caption{Details for all binary and continuous benchmark datasets.}
  \label{tab:details}
\end{table}

When evaluating the performance of \textproc{LearnRP}, we did not see a significant difference
between \textproc{SplitSID} and \textproc{SplitMax}, and so we only report figures for the latter.
When the domain is discrete, the vtree was learned by the top-down pairwise mutual information
algorithm discussed in \Cref{sec:learnpsdd}; when in the continuous, we replace mutual
information for Pearson's correlation. We use Gaussian mixture models as input nodes and fine-tune
the parameters of the resulting structure by standard batch EM. Experiments were carried out on a
single computer with a 12-core Intel i7 3.7GHz processor and 64GB RAM.

\textproc{LearnRP} was implemented in Julia\footnote{The source code can be found at
\url{https://github.com/RenatoGeh/RPCircuits.jl}.}, separately from \textproc{Strudel}'s and
\textproc{LearnPSDD}'s implementations found in the Juice package \citep{dang21}. For fairness,
when comparing runtimes for speed benchmarking we reran the same or similar setups as originally
reported for \textproc{XPC}, \textproc{Strudel} and \textproc{LearnPSDD} in the same machine used
for \textproc{LearnRP}, setting the number of iterations to 1000 for the two last \incrclass{}
algorithms. Similarly, we compare runtimes of a Rust implementation\footnote{Source code found at
\url{https://gitlab.com/marcheing/spn-rs}.} of \textproc{LearnSPN} on the same machine; we use
$k$-means for learning sums and G-test for products, setting $k=2$ for a shorter learning time. We
do not report running times for the original \textproc{Prometheus} code since, as far as we know,
the source was not made public and no working implementation was found.

\subsection{Binary data}

\Cref{tab:binary} shows benchmark results for binary data. For fairness, we report results for best
ensembles of \textproc{XPC}, \textproc{Strudel} and \textproc{LearnPSDD} so that all models
compared are smooth and structure decomposable (with the exception of \textproc{LearnSPN} and
\textproc{Prometheus} which are only decomposable) but nondeterministic. Columns
\textproc{LearnRP}-10, \textproc{LearnRP}-20, \textproc{LearnRP}-30 and \textproc{LearnRP}-F
correspond to runs of a single \textproc{LearnRP} circuit where we optimize the learned structure
with 10, 20, 30 and 100 iterations of batch EM with a batch size of 500 instances. After this, for
\textproc{LearnRP}-F, we proceed to run 30 iterations of full EM, a regularization technique
suggested in \citet{liu21}. The last two rows of \Cref{tab:binary} correspond to the average rank
of each algorithm, with the last one ignoring \textproc{LearnRP}-10, \textproc{LearnRP}-20 and
\textproc{LearnRP}-30.

\begin{table}[t]
  \resizebox{\textwidth}{!}{
  \begin{tabular}{c|cccc|cccc}
    \hline
    \textbf{Dataset} & \textbf{\textproc{LearnSPN}} & \textbf{\textproc{Strudel}} &
    \textbf{\textproc{LearnPSDD}} & \textbf{\textproc{XPC}} & \textbf{\textproc{LearnRP}-F} &
    \textbf{\textproc{LearnRP}-10} & \textbf{\textproc{LearnRP}-20} &
    \textbf{\textproc{LearnRP}-30}\\
    \hline
    \textsc{nltcs}     & 7m & 3m & 6m & 17s & 3m19s & 4s & 8s & 12s\\
    \textsc{plants}    & 50m & 41m & 26m & 1m3s & 26m & 52s & 1m47s & 2m32s\\
    \textsc{audio}     & 2h & 33m & 51m & 1m58s & 37m & 1m25s & 2m45s & 4m11s\\
    \textsc{jester}    & 52m & 24m & 37m & 1m20s & 29m & 1m20s & 2m37s & 3m54s\\
    \textsc{netflix}   & 1h & 14m & 33m & 2m8s & 46m & 1m40s & 3m22s & 4m48s\\
    \textsc{accidents} & 47m & 20m & 41m & 1m47s & 33m & 1m30s & 2m45s & 4m10s\\
    \textsc{book}      & >3h & 8m & 1h22m & 2m26s & 2h & 7m24s & 15m17s & 21m55s\\
    \textsc{dna}       & >3h & >3h & >3h & 17s & 11m & 1m27s & 2m47s & 3m13s\\
    \hline
  \end{tabular}
  }
  \caption{Learning time benchmark for a single circuit of \textproc{LearnSPN}, \textproc{Strudel},
    \textproc{LearnPSDD}, \textproc{XPC} and \textproc{LearnRP}.}
  \label{tab:bintime}
\end{table}

\Cref{tab:bintime} shows average running time for each learning algorithm to learn a \emph{single}
circuit for each dataset. Note that the performance reported in \Cref{tab:binary} for
\textproc{Strudel}, \textproc{LearnPSDD} and \textproc{XPC} correspond to ensemble results, meaning
that in all but \textproc{Strudel} (where a shared-structure mixture was learned) one has to
multiply the averages in \Cref{tab:bintime} with the number of components in the ensemble to get a
better picture of the speed difference.

\Cref{tab:binsize} shows circuit sizes for each learning algorithm. All except for
\textproc{LearnSPN} are reported as in their original works to better reflect the log-likelihood
results shown in \Cref{tab:binary}. Because \citet{gens13} do not report circuit sizes, we report
values from our own runs. We do not show circuit sizes for \textproc{Prometheus} since we could not
find the source code and \citet{jaini18a} do not report circuit sizes.

\begin{sidewaystable}
  \resizebox{\textheight}{!}{
  \begin{tabular}{c|ccccc|cccc}
    \hline
    \textbf{Dataset} & \textbf{\textproc{LearnSPN}} & \textbf{\textproc{Strudel}} &
    \textbf{\textproc{LearnPSDD}} & \textbf{\textproc{XPC}} & \textbf{\textproc{Prometheus}} &
    \textbf{\textproc{LearnRP}-F} & \textbf{\textproc{LearnRP}-10} &
    \textbf{\textproc{LearnRP}-20} & \textbf{\textproc{LearnRP}-30}\\
    \hline
    \textsc{accidents } & -30.03  & $|$-28.73$|$  & -30.16  & -31.02  & \textbf{-27.91}  & \underline{-28.71} & -30.18  & -29.66  & -29.17     \\
    \textsc{ad        } & -19.73  & \underline{-16.38}  & -31.78  & \textbf{-15.50}  & -23.96  & $|$-19.09$|$ & -22.51  & -20.87  & -20.37     \\
    \textsc{audio     } & -40.50  & -41.50  & \underline{-39.94}  & -40.91  & \textbf{-39.80}  & $|$-40.16$|$ & -41.05  & -40.66  & -40.68     \\
    \textsc{bbc       } & $|$-250.68$|$ & -254.41 & -253.19 & \textbf{-248.34} & \underline{-248.50}  & -255.16 & -259.17 & -260.32 & -254.09  \\
    \textsc{netflix   } & -57.02  & -58.69  & \textbf{-55.71}  & -57.58  & \underline{-56.47}  & $|$-56.97$|$  & -59.91  & -60.06  & -57.60    \\
    \textsc{book      } & -35.88  & -34.99  & -34.97  & -34.75  & $|$-34.40$|$  & \textbf{-33.51} & -36.15  & -34.45  & \underline{-33.95}     \\
    \textsc{20-newsgrp} & -155.92 & -154.47 & -155.97 & \underline{-153.75} & $|$-154.17$|$ & \textbf{-152.94} & -158.10 & -156.74 & -154.35   \\
    \textsc{reuters-52} & $|$-85.06$|$  & -86.22  & -89.61  & -84.70  & \textbf{-84.59}  & -85.56  & -91.61  & -87.73  & -86.62    \\
    \textsc{webkb     } & -158.20 & -155.33 & -161.09 & \textbf{-153.67} & $|$-155.21$|$ & \underline{-154.33} & -156.40 & -155.70 & -156.40   \\
    \textsc{dna       } & \textbf{-82.52}  & -86.22  & -88.01  & -86.61  & \underline{-84.45}  & $|$-85.12$|$  & -86.55  & -85.30  & -85.61    \\
    \textsc{jester    } & -75.98  & -55.03  & \textbf{-51.29}  & -53.43  & \underline{-52.80}  & $|$-52.97$|$  & -53.67  & -53.28  & -53.13    \\
    \textsc{kdd       } & -2.18   & $|$-2.13$|$   & \textbf{-2.11}   & -2.15   & \underline{-2.12}   & -2.13  & -2.20   & -2.19   & -2.15      \\
    \textsc{kosarek   } & -10.98  & -10.68  & \textbf{-10.52}  & -10.77  & \underline{-10.59}  & $|$-10.61$|$  & -10.97  & -10.86  & -10.85    \\
    \textsc{msnbc     } & \underline{-6.11}   & \textbf{-6.04}   & \textbf{-6.04}   & $|$-6.18$|$   & \textbf{-6.04}   & -6.34  & -6.39   & -6.42   & -6.39      \\
    \textsc{msweb     } & -10.25  & \textbf{-9.71}   & $|$-9.89$|$   & -9.93   & \underline{-9.86}   & $|$-9.89$|$  & -10.23  & -10.17  & -10.10     \\
    \textsc{nltcs     } & -6.11   & -6.06   & \textbf{-5.99}   & $|$-6.05$|$   & \underline{-6.01}   & -6.22  & -6.30   & -6.29   & -6.37      \\
    \textsc{plants    } & \underline{-12.97}  & $|$-12.98$|$  & -13.02  & -14.19  & \textbf{-12.81}  & -13.80  & -14.76  & -14.33  & -14.26    \\
    \textsc{pumsb-star} & $|$-24.78$|$  & \underline{-24.12}  & -26.12  & -26.06  & \textbf{-22.75}  & -26.45  & -27.14  & -26.78  & -26.75    \\
    \textsc{eachmovie } & -52.48  & -53.67  & -58.01  & -54.82  & $|$-51.49$|$  & \underline{-51.46}  & -52.68  & -52.34  & \textbf{-51.43}    \\
    \textsc{retail    } & -11.04  & \underline{-10.81}  & \textbf{-10.72}  & -10.94  & -10.87  & $|$-10.83$|$  & -11.08  & -10.98  & -10.95    \\
    \hline
    \multirow{2}{*}[-0.15em]{\textbf{Avg. Rank}}  & \textbf{4.3}     & $|$4.8$|$     & 5.2     & 4.85    & 4.9     & \underline{4.6}  & 5.95    & 5.45    & 4.95         \\
                                            & 4.3              & 3.65          & $|$3.55$|$ & 3.9  & \textbf{2.15} & \underline{3.45} &   &         &              \\
    \hline
  \end{tabular}
  }
  %[
    %-30.03   -28.73   -30.16   -31.02   -27.91   -30.18   -29.66   -29.17   -28.71  ;
    %-19.73   -16.38   -31.78   -15.50   -23.96   -22.51   -20.87   -20.37   -19.09  ;
    %-40.50   -41.50   -39.94   -40.91   -39.80   -41.05   -40.66   -40.68   -40.16  ;
    %-250.68  -254.41  -253.19  -248.34  -248.50  -259.17  -260.32  -254.09  -255.16 ;
    %-57.02   -58.69   -55.71   -57.58   -56.47   -59.91   -60.06   -57.60   -56.97  ;
    %-35.88   -34.99   -34.97   -34.75   -34.40   -36.15   -34.45   -33.95   -33.51  ;
    %-155.92  -154.47  -155.97  -153.75  -154.17  -158.10  -156.74  -154.35  -152.94 ;
    %-85.06   -86.22   -89.61   -84.70   -84.59   -91.61   -87.73   -86.62   -85.56  ;
    %-158.20  -155.33  -161.09  -153.67  -155.21  -156.40  -155.70  -156.40  -154.33 ;
    %-82.52   -86.22   -88.01   -86.61   -84.45   -86.55   -85.30   -85.61   -85.12  ;
    %-75.98   -55.03   -51.29   -53.43   -52.80   -53.67   -53.28   -53.13   -52.97  ;
    %-2.18    -2.13    -2.11    -2.15    -2.12    -2.20    -2.19    -2.15    -2.13   ;
    %-10.98   -10.68   -10.52   -10.77   -10.59   -10.97   -10.86   -10.85   -10.61  ;
    %-6.11    -6.04    -6.04    -6.18    -6.04    -6.39    -6.42    -6.39    -6.34   ;
    %-10.25   -9.71    -9.89    -9.93    -9.86    -10.23   -10.17   -10.10   -9.89   ;
    %-6.11    -6.06    -5.99    -6.05    -6.01    -6.30    -6.29    -6.37    -6.22   ;
    %-12.97   -12.98   -13.02   -14.19   -12.81   -14.76   -14.33   -14.26   -13.80  ;
    %-24.78   -24.12   -26.12   -26.06   -22.75   -27.14   -26.78   -26.75   -26.45  ;
    %-52.48   -53.67   -58.01   -54.82   -51.49   -52.68   -52.34   -51.43   -51.46  ;
    %-11.04   -10.81   -10.72   -10.94   -10.87   -11.08   -10.98   -10.95   -10.83  ;
  %]
  \caption{Performance of \textproc{LearnRP} in log-likelihood against state-of-the-art competitors
    in the twenty binary datasets for density estimation. Entries in \textbf{\textup{bold}}
    correspond to best performance, \underline{\textup{underlined}} entries are second best, and
    $|$\textup{barred}$|$ entries are third place.}
  \label{tab:binary}
\end{sidewaystable}

Overall, we found that \textproc{LearnRP} was competitive against the state-of-the-art, often
reaching second or first place in the case of \textproc{LearnRP}-F. As expected for such a simple
learning algorithm, it was hardly the best, although even under few EM iterations it was capable of
producing somewhat competitive models. Arguably, \textproc{LearnRP}'s strengths come from its
speed, learning circuits in a fraction of the time when compared to \textproc{LearnSPN},
\textproc{Strudel} and \textproc{LearnPSDD}. Perhaps the more direct competitor of
\textproc{LearnRP} in terms of scalability is \textproc{XPC}, showing the strengths of \randclass{}
when it comes to speed. Recall from \Cref{sec:xpcs}, however that \textproc{XPC} requires an
extensive grid search on the hyperparameters, while the performance of \textproc{LearnRP} mainly
depends on how much time one is willing to spend to fine-tune weights with parameter learning.
Notably, we found that \textproc{LearnRP} performs worse on data with fewer variables (e.g.\
\textsc{nltcs}, \textsc{msnbc}, \textsc{kdd}) and better on data with more variables (e.g.\
\textsc{20-newsgrp}, \textsc{book}, \textsc{webkb}), which is perhaps correlated with the smaller,
and respectively larger, circuits from \Cref{tab:binsize}.

\begin{table}[t]
  \resizebox{\textwidth}{!}{
  \begin{tabular}{c|cccc|cccc}
    \hline
    \textbf{Dataset} & \textbf{\textproc{LearnSPN}} & \textbf{\textproc{Strudel}} &
    \textbf{\textproc{LearnPSDD}} & \textbf{\textproc{XPC}} &
    \textbf{\textproc{LearnRP}-F} & \textbf{\textproc{LearnRP}-10} &
    \textbf{\textproc{LearnRP}-20} & \textbf{\textproc{LearnRP}-30}\\
    \hline
    \textsc{accidents } & 32708   & 75363   & 8418  & 11921   & 17609     & 17693   & 17309   & 17977   \\
    \textsc{ad        } & 40901   & 13152   & 12238 & 22093   & 89525     & 88877   & 94111   & 87117   \\
    \textsc{audio     } & 50130   & 55675   & 18208 & 29317   & 17319     & 17457   & 17369   & 17621   \\
    \textsc{bbc       } & 39389   & 29532   & 12335 & 14578   & 124439    & 124717  & 123765  & 126019  \\
    \textsc{netflix   } & 36286   & 27173   & 10997 & 39868   & 20663     & 20539   & 20631   & 20447   \\
    \textsc{book      } & 51493   & 54839   & 10978 & 13678   & 109881    & 108057  & 112303  & 109367  \\
    \textsc{20-newsgrp} & 119060  & 58749   & 15793 & 65881   & 486105    & 485427  & 485733  & 481613  \\
    \textsc{reuters-52} & 155191  & 36343   & 10410 & 36440   & 285060    & 286162  & 289292  & 285870  \\
    \textsc{webkb     } & 223847  & 25406   & 11033 & 17122   & 207462    & 209069  & 205128  & 208299  \\
    \textsc{dna       } & 12180   & 17507   & 3068  & 2616    & 26567     & 26527   & 26007   & 26495   \\
    \textsc{jester    } & 25076   & 27713   & 11322 & 20273   & 20679     & 20887   & 21025   & 21065   \\
    \textsc{kdd       } & 8755    & 6572    & 2915  & 13040   & 5389      & 5219    & 5229    & 5251    \\
    \textsc{kosarek   } & 19512   & 37583   & 7173  & 20938   & 32357     & 33455   & 33087   & 32993   \\
    \textsc{msnbc     } & 11606   & 20795   & 5465  & 4887    & 789       & 801     & 765     & 778     \\
    \textsc{msweb     } & 10743   & 2347    & 6581  & 12135   & 48545     & 48805   & 49715   & 50235   \\
    \textsc{nltcs     } & 1855    & 4373    & 1304  & 4401    & 515       & 511     & 491     & 519     \\
    \textsc{plants    } & 36596   & 119194  & 11583 & 13960   & 9335      & 9317    & 9941    & 9655    \\
    \textsc{pumsb-star} & 26206   & 108876  & 8298  & 8866    & 28725     & 28937   & 27731   & 29930   \\
    \textsc{eachmovie } & 54184   & 123996  & 20648 & 21369   & 69357     & 67047   & 68623   & 69193   \\
    \textsc{retail    } & 2158    & 3979    & 2989  & 6651    & 24345     & 24459   & 25031   & 24103   \\
    \hline
  \end{tabular}
  }
  \caption{Circuit size (in the number of nodes) comparison between \textproc{LearnRP} and the
    state-of-the-art in the twenty binary datasets for density estimation.}
  \label{tab:binsize}
\end{table}

\subsection{Continuous data}

\Cref{tab:cont} shows results for continuous data. Note that some entries are positive since the
log-likelihood of continuous variables can be positive. Because the evaluated continuous datasets
were small in size, we only show results for \textproc{LearnRP} under 100 iterations of EM. We also
do not run full EM after the batch variant as we found performance to degrade after doing so. When
constructing the circuit with \textproc{LearnRP}, we use mixtures of three Gaussians as input
nodes. We run learn both sum weights and input GMMs through EM. The last column of \Cref{tab:cont}
shows the circuit sizes of \textproc{LearnRP}.

\begin{table}[t]
  \resizebox{\textwidth}{!}{
  \begin{tabular}{c|cccccc|cc}
    \hline
    \textbf{Dataset} & \textbf{SRBMs} & \textbf{oSLRAU} & \textbf{GBMMs} & \textbf{GMMs} &
    \textbf{\textproc{Prometheus}} & \textbf{iSPTs} & \textbf{\textproc{LearnRP}} & \textbf{Size} \\
    \hline
    \textsc{abalone}    & -2.28  & -0.94  & -1.17  & ---   & -0.85   & ---   & -6.13  & 317   \\
    \textsc{ca}         & -4.95  & 21.19  & 3.42   & ---   & 27.82   & ---   & -5.84  & 2765  \\
    \textsc{quake}      & -2.38  & -1.21  & -3.76  & ---   & -1.50   & ---   & -3.76  & 79    \\
    \textsc{sensorless} & -26.91 & 60.72  & 8.56   & ---   & 62.03   & ---   & -38.46 & 12589 \\
    \textsc{banknote}   & -2.76  & -1.39  & -4.64  & ---   & -1.96   & ---   & -6.06  & 79    \\
    \textsc{flowsize}   & -0.79  & 15.32  & 5.72   & ---   & 18.03   & ---   & 2.20   & 49    \\
    \textsc{kinematics} & -5.55  & -11.13 & -11.20 & ---   & -11.12  & ---   & -11.02 & 319   \\
    \textsc{iris}       & ---    & ---    & ---    & -3.94 & -1.06   & -3.74 & -3.47  & 79    \\
    \textsc{oldfaith}   & ---    & ---    & ---    & -1.73 & -1.48   & -1.70 & -4.33  & 19    \\
    \textsc{chemdiabet} & ---    & ---    & ---    & -3.02 & -2.59   & -2.88 & -18.68 & 48    \\
    \hline
  \end{tabular}
  }
  \caption{Performance of \textproc{LearnRP} in log-likelihood against state-of-the-art competitors
    in ten continuous datasets for density estimation and function approximation. Entries in
    \textbf{\textup{bold}} correspond to best performance, \underline{\textup{underlined}} entries
    are second best, and $|$\textup{barred}$|$ entries are third place. Last column shows size (in
    the number of nodes) of circuits learned with \textproc{LearnRP}.}
  \label{tab:cont}
\end{table}

Strangely, we found that \textproc{LearnRP} behaves extremely poorly on continuous datasets. This
could be both due to the smaller number of variables in these datasets (as evidenced by the reduced
circuit sizes in \Cref{tab:cont}'s last column), causing \textproc{LearnRP} to produce very small
models; and because of the unpredictability and inconsistency of log-likelihood as a score in the
continuous.
