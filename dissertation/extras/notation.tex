\usepackage{amsthm}
\usepackage{bm}

\usetikzlibrary{shapes,arrows,positioning,fit,circuits.logic.US}

\newcommand{\bigo}{\mathcal{O}}
\newcommand{\set}[1]{\mathbf{#1}}
\DeclareMathOperator*{\argmin}{\normalfont{arg\,min}}
\DeclareMathOperator*{\argmax}{\normalfont{arg\,max}}
\DeclareMathOperator*{\Val}{\normalfont{Val}}
\DeclareMathOperator*{\Ch}{\normalfont{Ch}}
\DeclareMathOperator*{\Pa}{\normalfont{Pa}}
\DeclareMathOperator*{\Sc}{\normalfont{Sc}}
\DeclareMathOperator*{\Root}{\normalfont{Root}}
\DeclareMathOperator{\Sum}{\normalfont{S}}
\DeclareMathOperator{\Sums}{\mathbf{S}}
\DeclareMathOperator{\Prod}{\normalfont{P}}
\DeclareMathOperator{\Prods}{\mathbf{P}}
\DeclareMathOperator{\Leaf}{\normalfont{L}}
\DeclareMathOperator{\Leaves}{\mathbf{L}}
\DeclareMathOperator{\Node}{n}
\DeclareMathOperator{\Nodes}{\mathbf{N}}
\DeclareMathOperator{\Child}{\normalfont{C}}
\DeclareMathOperator{\Children}{\mathbf{C}}
\DeclareMathOperator{\Conj}{\otimes}
\DeclareMathOperator{\Disj}{\oplus}
\newcommand{\bs}[1]{\boldsymbol{#1}}
\newcommand{\defeq}{\vcentcolon=}
\DeclareMathOperator*{\ind}{\mathbbm{1}}
\DeclareMathOperator*{\True}{\normalfont{true}}
\DeclareMathOperator*{\False}{\normalfont{false}}
\DeclareMathOperator{\vtree}{\mathcal{V}}
\DeclareMathOperator{\Forget}{\normalfont{Forget}}
\newcommand{\ov}{\overline}
\newcommand{\tsup}{\textsuperscript}
\newcommand{\newGraphNode}[4]{\node[#4] (#1) at (#2) {\rotatebox{-90}{#3}}}
\newcommand{\newAndNode}[4]{\node[#4,and gate,fill=blue!50!red!30] (#1) at (#2) {\rotatebox{-90}{#3}}}
\newcommand{\newOrNode}[4]{\node[#4,or gate,fill=blue!50!green!30] (#1) at (#2) {\rotatebox{-90}{#3}}}

\newcommand{\newSumNode}[3][]{\node[#1,circle,draw,inner sep=0pt,minimum size=13pt,thick,fill=blue!50!green!30] (#2) at (#3) {$\bm{+}$}}
\newcommand{\newMixNode}[3][]{\node[#1,circle,draw,inner sep=0pt,minimum size=12pt,thick,fill=blue!50!green!30] (#2) at (#3) {$\sum$}}
\newcommand{\newProdNode}[3][]{\node[#1,circle,draw,inner sep=0pt,minimum size=12pt,thick,fill=blue!50!red!30] (#2) at (#3) {$\bm{\times}$}}
\newcommand{\newLeafNode}[3][]{\node[#1,circle,draw,inner sep=0pt,minimum size=12pt,thick,fill=orange!50!black!40] (#2) at (#3) {$\bm{\circ}$}}
\newcommand{\newCellNode}[3][]{\node[#1,circle,draw,inner sep=0pt,minimum size=12pt,thick,fill=orange!50!black!40] (#2) at (#3) {$\bm{\Box}$}}
\tikzset{sigmoid/.style={path picture={\begin{scope}[x=0.65pt,y=7pt] \draw plot[domain=-6:6](\x,{1/(1+exp(-1.5*\x))-0.5}); \end{scope}}}}
\tikzset{gaussian/.style={path picture={\begin{scope}[x=1pt,y=10pt] \draw plot[domain=-4:4](\x,{exp(-\x*\x*0.5)/2.5-0.1}); \end{scope}}}}
\newcommand{\newProjNode}[3][]{\node[#1,sigmoid,circle,draw,inner sep=2pt,minimum size=13pt,thick,fill=blue!50!green!30] (#2) at (#3) {};}
\newcommand{\newGaussNode}[3][]{\node[#1,gaussian,circle,draw,inner sep=2pt,minimum size=13pt,thick,fill=orange!50!black!40] (#2) at (#3) {};}
\newcommand{\newPartNode}[3][]{\node[#1,circle split,rotate=90,draw,inner sep=2pt,minimum size=12pt,thick,fill=blue!50!red!30] (#2) at (#3) {};}
\newcommand{\inode}[2][]{\tikz[baseline=-0.75ex]{#2[scale=0.8,#1]{r}{0,0};}}

\pgfmathdeclarefunction{gauss}{2}{%
  \pgfmathparse{1/(#2*sqrt(2*pi))*exp(-((x-#1)^2)/(2*#2^2))}%
}

\DeclareMathOperator{\gaussian}{\mathcal{N}}

\definecolor{LightGray}{gray}{0.85}
\definecolor{Red}{rgb}{0.75,0,0}
